%% LyX 2.1.0 created this file.  For more info, see http://www.lyx.org/.
%% Do not edit unless you really know what you are doing.
\documentclass[english]{article}
\usepackage[T1]{fontenc}
\usepackage[latin9]{inputenc}
\usepackage{color}
\usepackage{verbatim}
\usepackage{amsthm}
\usepackage{amsmath}
\usepackage{amssymb}
\usepackage{graphicx}
\usepackage{esint}

\makeatletter
%%%%%%%%%%%%%%%%%%%%%%%%%%%%%% Textclass specific LaTeX commands.
\theoremstyle{plain}
\newtheorem{thm}{\protect\theoremname}
  \theoremstyle{plain}
  \newtheorem{lem}[thm]{\protect\lemmaname}

%%%%%%%%%%%%%%%%%%%%%%%%%%%%%% User specified LaTeX commands.
% UAI 2015 template
\usepackage{proceed2e}

\usepackage{algorithmic}
\usepackage{algorithm}

\renewcommand{\algorithmicrequire}{\textbf{Input:}}
\renewcommand{\algorithmicensure}{\textbf{Output:}}

\usepackage{appendix}
\usepackage{amsmath}
\usepackage{amsfonts}
\usepackage{amsthm}
\usepackage{enumerate}
%\usepackage{etex}

    \usepackage[ plainpages = true, pdfpagelabels, 
                 pdfpagelayout = useoutlines,
                 bookmarks,
                 bookmarksopen = true,
                 bookmarksnumbered = true,
                 breaklinks = true,
                 linktocpage,
                 pagebackref,
                 colorlinks = true,
                 linkcolor = blue,
                 urlcolor  = blue,
                 citecolor = blue!50!black,
                 anchorcolor = green,
                 hyperindex = true,
                 hyperfigures
                 ]{hyperref} 
%\usepackage{index}
%\usepackage{longtable}

%\usepackage{mdwlist}
\usepackage{natbib}
\usepackage[resetlabels,labeled]{multibib}
%\usepackage{microtype}
\usepackage{soul}
\usepackage{subfig}
\usepackage{tikz}
\usepackage[textsize=small]{todonotes}

\usepackage{url}
%\usepackage[all]{xy}

%\usepackage{epstopdf}
%\usepackage{auto-pst-pdf}



\newcommand{\diag}{\mathop{\mathrm{diag}}}
\newcommand{\trace}{\mathop{\mathrm{tr}}}
\newcommand{\median}{\mathop{\mathrm{median}}}

\newcommand{\bx}{\mathbf{x}}				% all variables
\newcommand{\factor}{\psi}				% factor
\newcommand{\fis}[1]{\mathrm{ne}(#1)}   	% index set for variables connected to  factor
\newcommand{\fx}[1]{ \mathbf{x}_{\mathrm{ne}(#1)} }   	% variables of a factor
\newcommand{\xin}{\mathbf{x}_{ \mathrm{in} }} 			% parents of directed factor
\newcommand{\xout}{\mathbf{x}_{ \mathrm{out} }}			% child of directed factor
\newcommand{\msg}[2]{m_{#1 \rightarrow #2}}			% message from arg1 to arg2
\newcommand{\approxMsg}[3]{\hat{m}_{#1 \rightarrow #2}^{#3}}			% message from arg1 to arg2
\newcommand{\uncertaintyMsg}[3]{\hat{u}_{#1 \rightarrow #2}^{#3}}			% message from arg1 to arg2
\newcommand{\diffd}{\mathrm{d}}
\newcommand{\proj}{\mathrm{proj}}
\newcommand{\projP}[1]{\mathrm{proj} \left [ #1 \right]}
\newcommand{\argmin}[1]{\mathrm{arg}\mathrm{min}_{#1}}
\newcommand{\kld}[2]{\mathrm{KL} \left [ #1 || #2 \right ]}
\newcommand{\expectationE}[2]{ \mathbb{E}_{#2}  \left[ #1 \right] }


\newcommand{\wjnote}[1]{\textbf{\color{blue!80!black}{WJ: #1}} }
\newcommand{\aenote}[1]{\textbf{\color{red!50!black}{#1}} }
\newcommand{\nhnote}[1]{\textbf{\color{magenta!90!black}{#1}} }
\newcommand{\agnote}[1]{\textbf{\color{green!50!black}{#1}} }

\newcommand{\figref}[1]{Fig.~\ref{#1}}
\newcommand{\secref}[1]{Sec.~\ref{#1}}
\newcommand{\tabref}[1]{Table.~\ref{#1}}
%\newcommand{\eqref}[1]{Eq.~\ref{#1}}

\newcites{sup}{Supplementary}

\makeatother

\usepackage{babel}
  \providecommand{\lemmaname}{Lemma}
\providecommand{\theoremname}{Theorem}

\begin{document}

\title{Kernel-Based Just-In-Time Learning for Passing Expectation Propagation
Messages}
\author{}

\maketitle


\begin{abstract}
  We propose an efficient nonparametric strategy
  for learning a message operator in expectation propagation (EP),
  which takes as input the set of incoming messages to a factor node, and
  produces an outgoing message as output.
  This learned operator replaces the multivariate integral required in classical EP,
  which may not have an analytic expression.
  We use kernel-based regression, which is trained on
  a set of probability distributions
  representing the incoming messages, and the associated outgoing messages.
  The kernel approach has two main advantages: first, it is fast,
  as it is implemented using a novel two-layer random feature representation of the input message distributions;
  second, it has principled uncertainty estimates, and can be cheaply updated
  online, meaning it can request and incorporate new training data when it encounters inputs on
  which it is uncertain.  In experiments, our approach is able to solve learning problems
  where a single message operator is required for multiple, substantially different data sets (logistic regression for a variety of classification problems), where the ability to accurately assess uncertainty and to efficiently and robustly update
  the message operator are essential.
  \end{abstract}

\section{Introduction}

%% Intro paragraphs

%\cite{Rezende2014,Kingma2013,Stuhlmuller2013,Ross2011}


An increasing priority in Bayesian modelling is to make inference accessible and implementable for practitioners,
without their requiring specialist expertise in Bayesian inference. This
is a goal sought in probabistic programming languages \citep{WinGooStuSis11,GooManRoyBonTen08}, 
as well as in more granular, component-based procedures \citep{DuvLloGroetal13,GroSalFreTen12} and
systems \citep{SDT2014,Minka2014}. In all cases, the user
should be able to freely specify what they wish their model to express,
without having to deal with the complexities of sampling, variational
approximation, or distribution conjugacy.  In reality, however, model convenience and
simplicity can  limit
or undermine intended models, sometimes in ways the users
might not expect. To take one example, the inverted Gamma prior, which is a convenient
conjugate prior for the standard deviation, has quite pathological behaviour  \citep{Gelman2006}.
In general, more expressive, freely chosen models are more likely
to require expensive sampling or quadrature approaches, which can make
them challenging to implement or impractical to run.

We address the particular setting of expectation propagation \citep{Minka2001}, a message
passing algorithm wherein messages are confined to being members of a particular parametric
family. The process of integrating incoming messages over a factor potential, and projecting the result onto the
required output family, can be difficult, and in some cases not achieveable in closed form.
Thus, a number of approaches have been proposed to implement EP updates numerically, independent
of the details of the factor potential being used.  One approach, due to
\citet{Barthelme2011}, is to compute the message update via importance sampling.
While these estimates converge to the desired integrals for sufficient samples, the sampling procedure must be run at every iteration during inference,
hence  it is not viable for large-scale problems.

An improvement on this approach is to use importance sampled instances of input/output
message pairs to train a regression algorithm, which can then be used in place of the sampler.
The approach of \citet{Heess2013} uses neural networks to learn the mapping from incoming
to outgoing messages. The mappings learned performed well on a variety of practical problems,
however this approach comes with a disadvantage: it requires training data
that cover the entire set of possible input messages for a given type of problem,
and it has no way of assessing the uncertainty of its predictions (or of updating
the model online in the event that a prediction is uncertain). This can be a problem
if it is desired to learn a message operator for a broad class of problems:
to illustrate, if the goal is to do logistic regression, we would need to see
during training a set of representative messages for all classification problems
the user proposes to solve.

The disadvantages of the neural network approach were the basis for work by
\cite{Eslami2014}, who replaced the neural networks with random forests.
The random forests were updated online when messages were observed on which
the predictor was uncertain, where uncertainty was estimated via
the heuristic of looking at the variability of the forest predictions for each point \citep{CriSho13}.
The online updates were made by splitting the trees at their
leaves.
Both of these aspects are problematic, however. First, the heuristic
used in computing uncertainty has no guarantees: indeed, uncertainty estimation for
random forests remains a challenging topic of current research \cite{Hutter2009}. This is not merely a theoretical
consideration: in our experiments in Section \ref{sec:Experiments}, we demonstrate that such
uncertainty estimates become unstable and inaccurate as we move away from the initial
training data. Second, online updates of random forests may not work well
when the newly observed data is from a very different distribution to the
initial training sample \citep[e.g.][Fig. 3]{LakRoyTeh14}. Indeed, for large amounts of training set drift, the
leaf-splitting approach of Elsami et al. can result in a decision tree which is a long chain, giving a worst case
cost of prediction of $O(n)$ for training data of size $n$, vs the ideal of $O(\log(n))$
for balanced trees.

% AG: costs of GP are in paper ``Fast Gaussian Process Regression using KD-Trees'', which gives
% yet another way to speed up GP prediction.

We propose a novel, kernel-based approach to learning a message operator noparametrically
for expectation propagation. The learning algorithm takes the form of a distribution regression
problem, where the inputs are probabilty measures (represented as embeddings of the distributions to an RKHS), and the outputs are vectors of message
parameters \citep{Szabo2014}.
A first advantage of this approach is that one does not need to pre-specify customized features
of the distributions, as in \citep{Eslami2014,Heess2013}. Rather, 
we use a general characteristic kernel on input distributions 
\citep[eq. 9]{ChristmannSteinwart10}, which in our experiments gives better performance than customized features.
A potential downside of the kernel approach is that it can be computationally
costly, with training time of $O(n^3)$ and a cost of $O(n)$ to make a prediction. 
We address this problem by regressing directly in the primal from random features of the data \citep{Rahimi2007,Le2013,YanSmoZonWil14}.
We generalize this random feature representation to the case where the
the inputs are distributions, via a novel two-level random feature approach.
This gives us both fast prediction (linear in the number of
random features), and fast online updates (square in the number of random features).

A second advantage of our approach is that, being an instance of Gaussian process
regression, there are well established estimates of predictive uncertainty \citep[Ch. 2]{RasWil06}.
We use these uncertainty estimates so as to determine when to query the importance sampler
for additional input/output pairs. This is especially crucial if the message
operator is being trained on multiple different datasets, where the typical input messages
observed can be very different, and new training data are required.
We demonstrate empirically that when the input messge distributions are significantly different from
the initial training data, 
the uncertainty estimates returned for the kernel regression remain robust and informative,
whereas those for random forests are highly unstable.



Our paper proceeds as follows. In Section \ref{sec:EP}, we introduce the notation for expectation propagation,
and indicate how an importance sampling procedure can be used as an oracle to provide
``gold standard'' training data for the message operator (albeit at
significant computational cost).
We also give a brief overview of previous learning approaches to the problem, with a focus on
that of \citet{Eslami2014}, with brief details on the random forest regression,  uncertainty computation
procedure, and  overall  computational cost.
Next, in Section \ref{sec:Online}, we describe our kernel regression approach, and
the form of the kernel message operator mapping the input messages (probability distributions in an
RKHS) to output messages (sets of parameters of the outgoing messages), along with their uncertainty estimates.
For the purposes of fast and efficient representation, we provide a novel random feature-based implementation
for regression with distribution-valued inputs, which may be of independent interest.%: this has two layers, since features of the distribution
%embeddings are fed into features of the kernel on the embeddings.
Finally, in Section \ref{sec:Experiments}, we describe our experiments, which cover three topics. First, we show that our
uncertainty esimates are much more reliable than those obtained for random forests, hence more
useful in determining when to request additional training message pairs from the oracle.
Second, we demonstrate our method on artificial data with well-controlled statistical properties:
 we quickly learn the EP factor for logistic regression with high confidence,
 at a cost far lower than the importance sampling; and for the compound Gamma prior, again much faster than the
 quadrature implementation in Infer.net. Finally, we establish that our method
 is genuinely adaptive:
 we perform logistic regression on four different real-world datasets, updating our
 message operator as new datasets are made available, and the uncertainty over the resulting
  input
 messages increases.


%%%%%%%%%%%%%%%%%%%%%%%%%%%%%%%%%%%%%%%%%%%%%%%%%%%%%%%%%%%%%%%%%%%%%%%%%%%%%%%%%%%%%%%%%%





% \wjnote{add some text to link to ridge regression. Then...}
% By virtue of the primal form, it is straightforward to derive an update
% equation for an online-active learning during EP inference: if the
% predictive variance  on the current incoming messages is high, then we query the outgoing message from
% the importance sampler (oracle) and update the operator. Otherwise,
% the outgoing message is efficiently computed by the operator. Online
% learning of the operator lessens the need of the distribution on natural
% parameters of incoming messages used in training set generation. Determining
% an appropriate distribution was one of the unsolved problems in \cite{Heess2013}. 


\section{Background}


\label{sec:EP}

As is common in the message passing community we assume that distributions (or densities) over a set of variables $\bx = (x_1, \dots x_D)$ of interest can be represented as factor graphs, i.e.\
\begin{equation}
p(\bx) = \frac{1}{Z} \prod_{j=1}^J \factor_j(\fx{\factor_j}).
\end{equation}
The factors $\factor_j$ are non-negative functions which are defined over subsets $\fx{\factor_j}$ of the full set of variables $\bx$. These variables form the neighbors of the factor node $\factor_j$ in the factor graph, and we use $\fis{\factor_j}$ to denote the corresponding set of indices. $Z$ is the normalization constant. 

In this work we are specifically interested in dealing with models in which some of the factors have a non-standard form, or may not have a known analytic expression be known (i.e.\ "black box" factors). Although our approach applies to any such factor in principle, in this paper we focus on \textit{directed} factors $\factor(\xout | \xin)$ which specify a conditional distribution over variables $\xout$ given $\xin$ (and thus $\fx{\factor} = (\xout, \xin))$. The only assumption we make is that we are provided with a forward sampling function $f: \xin \mapsto \xout$, i.e. a function that maps (stochastically or deterministically) a setting of the input variables $\xin$ to a sample from the conditional distribution over $\xout \sim \factor(\cdot| \xin)$. One very natural way of specifying $f$ is a short piece of code in a probabilistic program.

%--------------------------------------------------------------------

\subsection{Message passing EP}
\label{sec:EP:MP}

The belief propagation - or sum-product algorithm - computes marginal distributions over subsets of variables by iteratively passing messages between variables and factors, ensuring consistency of the obtained marginals at convergence. Specifically, the messages sent from a factor $\factor$ to variable $x_i$ (where $i \in \fis{\factor}$) are computed as
\begin{equation}
\msg{ \factor }{i}(x_i) = 
\int \factor (\fx{\factor}) \prod_{i' \in \fis{\factor} \textbackslash i} \msg{i'}{\factor}(x_{i'}) \diffd \bx_{\fis{\factor} \textbackslash i},
\label{eq:msgPassing:BP}
\end{equation}
where $\msg{i'}{\factor}(\cdot)$ are the messages sent to factor $\psi$ from its neighboring variables $x_{i'}$ other than $x_i$.

EP introduces an approximation in the case when the message $\msg{ \factor }{i}(\cdot)$ 
does not have a simple parametric form, by projecting the exact 
marginal $\msg{ i }{\factor}(x_i)\msg{ \factor }{i}(x_i)$ onto a member of 
some class of known parametric distributions. 
That is $\msg{ \factor }{i}(x_i)$ is given by
%
\begin{equation}
\small
\frac{\projP{ \msg{i}{\factor}(x_{i})
\int \factor (\fx{\factor}) \prod_{i' \in \fis{\factor} \textbackslash i} \msg{i'}{\factor}(x_{i'}) \diffd \bx_{\fis{\factor} \textbackslash i}}}
{\msg{i}{\factor}(x_{i})}
\label{eq:msgPassing:EP}
\end{equation}
%
where $\projP{p} = \argmin{q \in \mathcal{Q}} \kld{p}{q}$, and $\mathcal{Q}$ is typically in the exponential family, e.g.\ the set of Gaussian or Beta distributions.

%--------------------------------------------------------------------

\subsection{Monte-Carlo message approximation}
\label{sec:EP:MC}

By projecting messages onto simple parametric forms, EP introduces an approximation that allows message passing to proceed when the true messages are not easily representable in closed form.
Computing \textit{the numerator} of (\ref{eq:msgPassing:EP}) remains, however, a challenging problem in itself as it requires evaluating a high-dimensional integral as well as minimization of the Kullback-Leibler divergence to some non-standard distribution. For non-trivial factors with known analytic form this often requires hand-crafted approximations \cite{XYZ} or, for instance, the use of expensive numerical integration techniques; for "black-box" factors as described above (sec.\ \ref{sec:EP}) it is entirely unclear how to proceed. 

%For such factors the only assumption we make is that we are provided with a forward sampling function $f: \xin \mapsto \xout$, i.e. a function that maps (stochastically or deterministically) a setting of the input variables $\xin$ to a sample from the conditional distribution over $\xout \sim \factor(\cdot| \xin)$. Such a function may, for instance, be

This significantly limits the ease with which EP can be applied and its scope in practice. 
%However, while deterministic approximations to (\ref{eq:msgPassing:EP}) may be hard to come by,
Motivated by these shortcomings \cite{Barthelme2011,Heess2013,Eslami2014} exploit an alternative, stochastic approximation: In general, the projection $\projP{\cdot}$ of $p$ onto some member of the exponential family corresponds to matching the relevant moments of $p$, e.g.\ its mean and variance, if $q$ is required to be a Gaussian distribution. Thus, projecting $p$ onto a general member of the exponential family, $q(x|\eta)=h(\theta)\exp\left(\eta^{\top}u(x)-A(\eta)\right)$ with a vector of sufficient statistics $u$ and natural parameters $\eta$ requires us to compute the expectation of $u(\cdot)$ under the numerator of \ref{eq:msgPassing:EP}).

A sample based approximation of this expectation can, in general, obtained via MCMC. Given a forward-sampling function $f$ as described above, one especially simple approach is importance sampling:
\begin{align}
\expectationE{u(x)}{\fx{\factor}\sim b }
&\approx \frac{1}{N} \sum_{l=1}^N w(\fx{\factor}^l) u(x_i^l),~~~~\fx{\factor}^s \sim \tilde{b}
\label{eq:msgIS}
\end{align}
where, on the left hand side, 
\begin{equation}
b(\fx{\factor}) = \factor (\fx{\factor}) \prod_{i \in \fis{\factor}} \msg{i}{\factor}(x_{i}). 
\end{equation}
On the right hand side we draw samples $\fx{\factor}^l$ from some proposal distribution $\tilde{b}$ which we choose to be 
\begin{equation}
\tilde{b}(\fx{\factor}) = r(\xin)\factor(\xout | \xin)
\end{equation}
for some $r$ with appropriate support, and compute importance weights 
\begin{equation}
w(\fx{\factor}) = \frac{\prod_{i \in \fis{\factor}} \msg{i}{\factor}(x_{i})}{r(\xin)}.
\end{equation}

The thus estimated expected sufficient statistics provide us with an estimate of the parameters $\eta$ of the result $q$ of the projection $\projP{p}$, from which the message is readily computed.


%--------------------------------------------------------------------

\subsection{Just-in-time learning of messages}
\label{sec:EP:JIT}

Message approximations as in the previous section could be used directly when running the EP algorithm as in \cite{Barthelme2011}, but this approach can suffer from the unreliability of the importance sampling estimates when the number of samples $N$ is not very large. On the other hand, for large $N$ the computational cost of running EP with approximate messages can be very high, as the message approximations have to be re-computed in each iteration of EP. To obtain low-variance message approximations at lower computational cost \cite{Heess2013,Eslami2014} both amortize previously computed approximate messages by training a function approximator to directly map a tuple of incoming variable-to-factor messages $(\msg{i'}{\factor} )_{i' \in \fis{\factor}}$ to an approximate factor to variable message $\msg{\factor}{i}$, i.e.\ they learn a mapping
\begin{equation}
\approxMsg{\factor}{i}{\theta}: (\msg{i'}{\factor} )_{i' \in \fis{\factor}} \mapsto \msg{\factor}{i},
\end{equation}
where $\theta$ are the parameters of the function approximator. Note that for exponential family distribution this effectively reduces to a multi-variate regression problem as each message can be represented by a finite-dimensional parameter vector, and the message update (\ref{eq:msgPassing:EP}) can thus be understood as a vector valued function.

While \cite{Heess2013} use neural networks and a very large, fixed training set to learn their approximate message operator prior to running EP with the intractable factor, \cite{Eslami2014} investigate an online learning approach in which the approximate message operator is updated on the fly, during inference, depending on whether the function approximator can be expected to produce a reliable message approximation. To this end, they endow their function approximator with an uncertainty estimate which  
\begin{equation}
\uncertaintyMsg{\factor}{i}{\theta}: (\msg{i'}{\factor} )_{i' \in \fis{\factor}} \mapsto u,
\end{equation}
where $u$ indicates the expected unreliability  of the predicted, approximate message $\msg{\factor}{i}$ returned by $\approxMsg{\factor}{i}{\theta}$. If $u = \uncertaintyMsg{\factor}{i}{\theta} \left( (\msg{i'}{\factor} )_{i' \in \fis{\factor}}\right)$ exceeds a pre-defined threshold, the required message is approximated via importance sampling (cf.\ eq.\ \ref{eq:msgIS}) and $\approxMsg{\factor}{i}{\theta}$ is re-trained on this new datapoint (leading to a new set of parameters  $\theta'$ with $\uncertaintyMsg{\factor}{i}{\theta'} \left( (\msg{i'}{\factor} )_{i' \in \fis{\factor}}\right)) < \uncertaintyMsg{\factor}{i}{\theta} \left( (\msg{i'}{\factor} )_{i' \in \fis{\factor}}\right)$.

The just-in-time approach of \cite{Eslami2014} critically relies on two properties of the class of function approximators used to represent $\approxMsg{\factor}{i}{\theta}$: they need to provide  (un)reliability estimates $\uncertaintyMsg{\factor}{i}{\theta}$ over their domain, and they need to allow for online learning. \cite{Eslami2014} use decision trees. [ These do fulfill both criteria, however, only when resorting to unprincipled heuristics. In the next section we present a framework based on kernel mean embeddings which possesses the desired properties and achieves this in a principled manner. ]
\nhnote{Ali: if you could fill in some sentences describing, at a high level, online learning and uncertainty estimates in trees to justify the claim that these are heuristics/can be improved upon, that would be great. Also, we may want to move the computational complexity of tree operations here.}



%---------------------------------------------------------------------------
% NH: commented out the text below because it's effectively contained in section 2

%\section{Expectation Propagation (EP)\label{sec:Expectation-Propagation-(EP)}}
%\wjnote{The content in this section was largely described in the previous two.
%We may remove this section.}
%
%Expectation propagation \citep{Minka2001,Bishop2006} (EP) is a commonly
%used approximate inference method for inferring the posterior distribution
%of latent variables given observations. 
%%
%%\begin{comment}
%%In a typical directed graphical
%%model, the joint distribution of the data $X=\left\{ X_{1},\ldots,X_{n}\right\} $
%%and latent variables $\theta=\left\{ \theta_{1},\ldots,\theta_{t}\right\} $
%%takes the form of a product of factors, $p(X,\theta)=\prod_{i=1}^{m}f_{i}(X|\theta)$
%%where each factor $f_{i}$ may depend on only a subset of $X$ and
%%$\theta$. With $X$ observed, EP approximates the posterior with
%%$q(\theta)\propto\prod_{i=1}^{m}m_{f_{i}\rightarrow\theta}(\theta)$
%%where $m_{f_{i}\rightarrow\theta}$ is an approximate factor corresponding
%%to $f_{i}$ with the constraint that it has a chosen parametric form
%%(e.g., Gaussian) in the exponential family (ExpFam). EP takes into
%%account the fact that the final quantity of interest is the posterior
%%$q(\theta)$ which is given by the product of all approximate factors.
%%In finding the $i^{th}$ approximate factor $m_{f_{i}\rightarrow\theta}$,
%%EP uses other approximate factors $m_{\theta\rightarrow f_{i}}(\theta):=\prod_{j\neq i}m_{f_{j}\rightarrow\theta}(\theta)$
%%as a context to determine the plausible range of $\theta$. EP iteratively
%%refines $m_{f_{i}\rightarrow\theta}$ for each $i$ with \wjnote{No need to mention this special-case msg.}$m_{f_{i}\rightarrow\theta}(\theta)=\frac{\text{proj}\left[\int dX\, f(X|\theta)m_{X\rightarrow f_{i}}(X)m_{\theta\rightarrow f_{i}}(\theta)\right]}{m_{\theta\rightarrow f_{i}}(\theta)}$
%%where $\text{proj}\left[r\right]=\arg\min_{q\in\text{ExpFam}}\text{KL}\left[r\,\|\, q\right]$
%%and $m_{X\rightarrow f_{i}}(X):=\delta(X-X_{0})$ if $X$ is observed
%%to be $X_{0}$. In the EP literature, $m_{\theta\rightarrow f_{i}}$
%%is known as a cavity distribution. \aenote{AE: Too many inline equations.} 
%%
%%The projection can be carried out by the following moment matching
%%procedure. Assume an ExpFam distribution $q(\theta|\eta)=h(\theta)\exp\left(\eta^{\top}u(\theta)-A(\eta)\right)$
%%where $u(\theta)$ is the sufficient statistic\textbf{ }of $q$, $\eta$
%%is the natural parameter and $A(\eta)=\log\int d\theta\, h(\theta)\exp\left(\eta^{\top}u(\theta)\right)$
%%is the log-partition function. It can be shown that $q^{*}=\text{proj}\left[r\right]$
%%satisfies $\mathbb{E}_{q^{*}(\theta)}\left[u(\theta)\right]=\mathbb{E}_{r(\theta)}\left[u(\theta)\right].$
%%That is, the projection of $r$ onto ExpFam is given by $q^{*}\in\text{ExpFam}$
%%that has the same moment parameters as the moments under $r$. 
%%
%%\wjnote{Should jump to this general msg from the beginning.}
%%\end{comment}
%
%Under an approximation that each factor fully factorizes, an EP message
%from a factor $f$ to a variable $V$ takes the form 
%\begin{align}
%m_{f\rightarrow V}(v) & =\frac{\text{proj}\left[\int d\mathcal{V}\backslash\{v\}\, f(\mathcal{V})\prod_{V'\in\mathcal{V}}m_{V'\rightarrow f}(v')\right]}{m_{V\rightarrow f}(v)}\nonumber \\
% & :=\frac{\text{proj}\left[r_{f\rightarrow V}(v)\right]}{m_{V\rightarrow f}(v)}:=\frac{q_{f\rightarrow V}(v)}{m_{V\rightarrow f}(v)}\label{eq:general_ep_msg}
%\end{align}
%where $\mathcal{V}=\mathcal{V}(f)$ is the set of variables connected
%to $f$ in the factor graph. In the previous case of $m_{f_{i}\rightarrow\theta}$,
%we have $\mathcal{V}(f)=\left\{ X,\theta\right\} $ and $V$ in Eq.
%\ref{eq:general_ep_msg} corresponds to $\theta$. Typically, when
%the factor $f$ is complicated, the integral defining $r_{f\rightarrow V}$
%becomes intractable. Quadrature rules or other numerical integration
%techniques are often applied to approximate the integral. \aenote{AE: This is the main point we want to get accross. Either needs more emphasis or should be mentioned sooner.}
%
%
%
%\section{Learning to Pass EP Messages\label{sec:Learning-to-Pass}}
%
%Our goal is to learn a message operator $C_{f\rightarrow V'}$ with
%signature $\left[m_{V\rightarrow f}\right]_{V\in\mathcal{V}(f)}\mapsto q_{f\rightarrow V'}$
%which takes in all incoming messages $\left\{ m_{V\rightarrow f}\mid V\in\mathcal{V}(f)\right\} $
%and outputs $q_{f\rightarrow V'}(v')$ i.e., the numerator of Eq.
%\ref{eq:general_ep_msg}. For inference, we require one such operator
%for each recipient variable $V'\in\mathcal{V}(f)$ i.e., in total
%$|\mathcal{V}(f)|$ operators need to be learned for $f$. Operator
%learning is cast as a distribution-to-distribution regression problem
%where the training set $S_{V'}:=\{([m_{V\rightarrow f}^{n}]_{V\in\mathcal{V}(f)},q_{f\rightarrow V'}^{n})\}_{n=1}^{N}$containing
%$N$ incoming-outgoing message pairs can be generated as in \cite{Heess2013,Eslami2014}
%by importance sampling to compute the mean parameters $\mathbb{E}_{r_{f\rightarrow V'}(v')}\left[u(v')\right]$
%for moment matching. In principle, the importance sampling itself
%can be used in EP for computing outgoing messages \citep{Barthelme2011}.
%The scheme is, however, expensive as we need to draw a large number
%of samples for each outgoing message to be sent. In our case, the
%importance sampling is used for data set generation which is done
%offline before the actual inference.
%
%The assumptions needed for the generation of a training set are as
%follows. First, we assume the factor $f$ takes the form of a
%conditional distribution $f(v_{1}|v_{2},\ldots,v_{|\mathcal{V}(f)|})$.
%Second, given $v_{2},\ldots,v_{|\mathcal{V}(f)|}$, $v_{1}$ can
%be sampled from $f(\cdot|v_{2},\ldots,v_{|\mathcal{V}(f)|})$. 
%The ability to evaluate $f$ is not assumed. 
%In contrast to \cite{Heess2013}, we do not assume that a distribution on the natural parameters of all incoming messages is available. \wjnote{because we use onlne-active learning... More text here.}
%
%%Finally we assume that a distribution on the natural parameters of all incoming messages $\left\{ m_{V\rightarrow f}\right\} _{V\in\mathcal{V}(f)}$
%%is available. The distribution is used solely to give a rough idea
%%of incoming messages the learned operator will encounter during the
%%actual EP inference. In practice, we only need to ensure that the
%%distribution sufficiently covers the relevant region in the space
%%of incoming messages. 
%%\aenote{AE: This sentence needs reworking now that you only train on messages observed in the model.}
%
%
%\end{comment}
%---------------------------------------------------------------------------




\section{Proposed Method}\label{sec:ADD}

In principle, any distribution-to-vector regression function can be applied to learn a message operator $C_{f\rightarrow V'}$, once the training set $S_{V'}$ is obtained.
Given incoming messages, the operator outputs $q_{f\rightarrow V'}$
from which the outgoing EP message is given by $m_{f\rightarrow V'}=q_{f\rightarrow V'}/m_{V'\rightarrow f}$
which can be computed analytically. We opt for kernel ridge regression
\citep{Scholkopf2002} as our message operator for its simplicity,
its ease of use in an online setting during inference (see \secref{sec:kernel_ep_online}), and rich supporting theory. 

\subsection{Kernel Ridge Regression}
\wjnote{We need to unify Kernel ridge regression and Bayesian linear regression sections. 
Posterior mean estimate of the Bayesian regression is the same as the solution to primal kernel ridge regression.
Bayesian regression assumes a Gaussian noise output allowing us to compute predictive variance.}


%\aenote{AE: Too many inline equations in this section. Pull out the most important ones?} 
We consider here the problem of regressing smoothly from distribution-valued
inputs to feature-valued outputs. We follow the regression framework
of \cite{Micchelli2005}, with convergence guarantees provided by
\cite{Caponnetto2007}. Under smoothness constraints, this regression
can be interpreted as computing the conditional expectation of the
output features given the inputs \citep{Gruenewaelder2012}.

Let $\mathsf{X}=\left(\mathsf{x}_{1}|\cdots|\mathsf{x}_{N}\right)$
be the training regression inputs and $\mathsf{Y}=\left(\mathbb{E}_{q_{f\rightarrow V'}^{1}}u(v')|\cdots|\mathbb{E}_{q_{f\rightarrow V'}^{N}}u(v')\right)\in\mathbb{R}^{D_{y}\times N}$
be the regression outputs. The ridge regression in the primal form
seeks $\mathsf{W}\in\mathbb{R}^{D_{y}\times D}$ for the regression
function $g(\mathsf{x})=\mathsf{W}\mathsf{x}$ which minimizes the
squared-loss function $J(\mathsf{W})=\sum_{i=1}^{N}\|\mathsf{y}_{i}-\mathsf{W}\mathsf{x}_{i}\|_{2}^{2}+\lambda\trace\left(\mathsf{W}\mathsf{W}^{\top}\right)$
where $\lambda$ is a regularization parameter and $\trace$ denotes
a matrix trace. 

It is well known that the solution is given by $\mathsf{W}=\mathsf{Y}\mathsf{X}^{\top}\left(\mathsf{X}\mathsf{X}^{\top}+\lambda I\right)^{-1}$
which has an equivalent dual solution $\mathsf{W}=\mathsf{Y}\left(K+\lambda I\right)^{-1}\mathsf{X}^{\top}=A\mathsf{X}^{\top}$
. The dual formulation allows one to regress from any type of input
objects if a kernel can be defined between them. All the inputs enter to the regression
function through the gram matrix $K\in\mathbb{R}^{N\times N}$ where
$\left(K\right)_{ij}=\kappa(\mathsf{x}_{i},\mathsf{x}_{j})$ yielding
the regression function of the form $g(\mathsf{x})=\sum_{i=1}^{N}a_{i}\kappa(\mathsf{x}_{i},\mathsf{x})$
where $A:=(a_{1}|\cdots|a_{N})$. The dual formulation therefore allows
one to straightforwardly regress from incoming messages to vectors
of mean parameters.
Although this property is appealing, the training
size $N$ in our setting can be chosen to be arbitrarily large, making
computation of $g(\mathsf{x})$ expensive for a new unseen point $\mathsf{x}$.


To eliminate the dependency on $N$, we propose to apply random Fourier
features \citep{Rahimi2007} $\hat{\phi}(\mathsf{x})\in\mathbb{R}^{D}$
for $\mathsf{x}:=[m_{V\rightarrow f}]_{V\in\mathcal{V}(f)}$ such
that $\kappa(\mathsf{x},\mathsf{x}')\approx\hat{\phi}(\mathsf{x})^{\top}\hat{\phi}(\mathsf{x}')$
where $D$ is the number of random features. The use of the random
features allows us to go back to the primal form of which the regression
function $g(\hat{\phi}(\mathsf{x}))=\mathsf{W}\hat{\phi}(\mathsf{x})$
can be computed efficiently. In effect, computing an EP outgoing message
requires nothing more than a multiplication of a matrix $\mathsf{W}$
($D_{y}\times D$ ) with the $D$-dimensional feature vector generated
from the incoming messages. 


\subsection{Bayesian Linear Regression}

Denote $X=\left(x_{1}|\cdots|x_{N}\right)\in\mathbb{R}^{D\times N}$
as (finite-dimensional) input and $Y=\left(y_{1}|\cdots|y_{N}\right)\in\mathbb{R}^{1\times N}$
(one coordinate of all expected sufficient statistics) as regression
output. In our application, $x_{n}$ will denote a random feature
vector for $[m_{V\rightarrow f}^{n}]_{V\in\mathcal{V}(f)}$. Assume
%
\begin{align*}
w & \sim\mathcal{N}\left(w;0,I_{D}\sigma_{0}^{2}\right)\\
Y\mid X,w & \sim\mathcal{N}\left(Y;w^{\top}X,\sigma_{y}^{2}I_{N}\right).
\end{align*}
%
The output noise variance $\sigma_{y}^{2}$ should capture the intrinsic
stochasticity of the importance sampler used to generate $Y$. It
follows that the posterior of $w$ is given by \citep{Bishop2006}
\begin{align*}
p(w\mid Y) & =\mathcal{N}(w;\mu_{w},\Sigma_{w})\\
\Sigma_{w} & = \left( XX^{\top}\sigma_{y}^{-2}+\sigma_{0}^{-2}I \right)^{-1} \\
\mu_{w} & =\Sigma_{w}XY^{\top}\sigma_{y}^{-2}.
%=\left(XX^{\top}+\frac{\sigma_{y}^{2}}{\sigma_{0}^{2}}I\right)^{-1}XY^{\top}.
\end{align*}
The noise variance $\sigma_{y}^{2}$ is proportional to the regularization
parameter in linear regression. 
The predictive distribution on the output $y^{*}$ given an observation $x^{*}$ is
%
\begin{align*}
p(y^{*}|x^{*},Y) & =\int dw\, p(y^{*}|w,x^{*},Y)p(w|Y)\\
 & =\mathcal{N}\left(y^{*};x^{*\top}\mu_{w},x^{*\top}\Sigma_{w}x^{*}+\sigma_{y}^{2}\right)
% & =\mathcal{N}(y^{*};m^{*},v^{*}).
\end{align*}

% The marginal likelihood is given by
%\begin{align*}
%p(Y|X) & =\int dw\, p(Y|X,w)p(w)\\
% & =\mathcal{\mathcal{N}}\left(Y;0,K\sigma_{0}^{2}+\sigma_{y}^{2}I_{N}\right)\\
% & =\det\left(2\pi\Sigma\right)^{-1/2}\exp\left(-\frac{1}{2}Y^{\top}\left(K\sigma_{0}^{2}+\sigma_{y}^{2}I_{N}\right)^{-1}Y\right)
%\end{align*}
%where $K=X^{\top}X$ is a gram matrix of $X$. 
%One can do gradient ascent on $\log p(Y|X)$
%to optimize $\sigma_{0}^{2},\sigma_{y}^{2}$ and kernel parameters.

For simplicity, we treat each output (expected sufficient statistic) as a separate regression problem. 
Treating all outputs jointly can be achieved with a multi-output kernel \citep{Alvarez2011}.

\subsection{Online Update \label{sec:kernel_ep_online}}
 
In this section, we consider an online update for $\Sigma_{w}$ and
$\mu_{w}$ when observations $X$ comes in stream. We will use $\cdot^{(n)}$
to denote a quantity constructed from $n$ samples. Recall that $\Sigma_{w}^{-1(N)}=XX^{\top}\sigma_{y}^{-2}+\sigma_{0}^{-2}I$.
The posterior covariance matrix at time $N+1$ can be written as 
\begin{align*}
\Sigma_{w}^{(N+1)} & =\left(XX^{\top}\sigma_{y}^{-2}+ \sigma_{0}^{-2}I  
 + x_{N+1}x_{N+1}^{\top}\sigma_{y}^{-2} \right)^{-1}\\
 & =\left(\Sigma_{w}^{-1(N)}+x_{N+1}x_{N+1}^{\top}\sigma_{y}^{-2}\right)^{-1}\\
 & =\Sigma_{w}^{(N)}-\frac{\Sigma_{w}^{(N)}x_{N+1}x_{N+1}^{\top}\Sigma_{w}^{(N)}\sigma_{y}^{-2}}{1+x_{N+1}^{\top}\Sigma_{w}^{(N)}x_{N+1}\sigma_{y}^{-2}}
\end{align*}
where we used the Sherman-Morrison formula. 
In this form, the posterior covariance at time $N+1$ can be expressed in terms of 
the covariance at time $N$.
%\[
%\left(A+uv^{\top}\right)^{-1}=A^{-1}-\frac{A^{-1}uv^{\top}A^{-1}}{1+v^{\top}A^{-1}u}
%\]
%with $A=\Sigma_{w}^{-1(N)}$. 
Updating $\Sigma_{w}$ costs $O(D^{2})$ per one new observation. 
For $\mu_{w}=\Sigma_{w}XY^{\top}\sigma_{y}^{-2}$, we maintain
$XY^{\top}\in\mathbb{R}^{D}$ and update it with
\[
\left(XY^{\top}\right)^{(N+1)}=\left(XY^{\top}+x_{N+1}y_{N+1}\right), 
\]
costing $O(D)$ per update. 
%In the first iteration when no data are observed, $\Sigma_{w}^{(1)}=\sigma_{0}^{2}I$.

\wjnote{Mention initial minibatch training here.}

\subsection{Online-Active Learning}

%If the predictive variance of the current incoming messages $x^{*}$
%is high, the operator queries the correct outgoing message from the importance sampler
%(oracle) and update the $\Sigma_{w}$ and $\mu_{w}$. 
%Otherwise, the outgoing message is efficiently computed by the operator. 

Since we have $D_{y}$ regression functions, 
for each tuple of incoming messages $x^{*}$, there are $D_{y}$
predictive variances, $v_{1}^{*},\ldots,v_{D_{y}}^{*}$, one for each
output. 
Let $\{\tau_{i}\}_{i=1}^{D_{y}}$ be pre-specified predictive variance thresholds.
\wjnote{How should we choose these thresholds ?}
Given a new input $x^{*}$, if $v_{1}^{*}>\tau_{1}$ or $\cdots$
or $v_{D_{y}}^{*}>\tau_{D_{y}}$ i.e., the operator is uncertain, 
a query is made to the oracle to obtain a ground truth $y^{*}$. 
The pair $(x^{*},y^{*})$ is then
used to update $\Sigma_{w}$ and $\mu_{w}$ as described previously. 
%Querying the oracle incurs a high computational cost.


\subsection{Computational Complexity}
\wjnote{Compare ridge regression cost with Ali's forest here ?}

Let:
\begin{itemize}
\item $K$ be the number of trees in the random forest,
\item $D_t$ be the dimensionality of incoming message group representation for tree traversal,
\item $D_r$ be the dimensionality of incoming message group representation for regression,
\item $N$ be the number of oracle consultations so far, and
\item $M$ be number of training points in each leaf.
\end{itemize}

Assuming that the depth of trees is $O(\log(N))$, one prediction from a random forest costs $O(K D_r D_t \log(N))$, and one update costs $O(K D_r^3 D_t \log(N))$.

This is because for each tree in forest $O(K)$, tree traversal involves $O(\log(N))$ steps where each costs $O(D_t)$ (splitting on an internal node involves a linear regression on the tree representation of the incoming message group), and one prediction costs $O(D_r)$. 

One full training of a regression function at a leaf costs $O(D_r^2 (D_r + M))$. 
However typically $M$ is negligible at a lower depth, therefore training at a leaf costs $O(D_r^3)$.
Therefore the total updating cost across all trees is $O(K D_r^3 D_t log(N))$.

Concrete numbers for forests:

\begin{itemize}
\item $K=64$ always.
\item $D_r$ depends on the number of incoming messages to the factor. In most of my experiments $D=14$. Why? Gaussian, Beta and Gamma distributions are all parameterised by two numbers, most factors had two incoming messages and we used quadratic regressors, so the regression features were $a$, $b$ (parameters of first incoming message), $c$, $d$ (parameters of second incoming message), $a^2$, $b^2$, $c^2$, $d^2$, $ab$, $ac$, $ad$, $bc$, $bd$ and $cd$.
\item $D_t$ also depends on the number of incoming messages to the factor. This was typically in the order of tens (10 to 20). For each incoming message there would be ~5 parameters (e.g. for a Gaussian there would be the mean, variance, mean $\times$ precision, precision and mode) and there would also be the factor evaluated at the mean and mode of the incoming messages (each a single number).
\item $N$ in the order of thousands (1,000 to 5,000).
\item $M$ roughly between 10 and 50 in each leaf.
\end{itemize}

\section{Kernels on Distributions}

A number of kernels on distributions have been studied in the literature
\citep{Jebara2003,Jebara2004}. Relevant to us are kernels whose random
features can be efficiently computed. Due to the space constraint,
we only give a few examples here.

\subsection{Expected Product Kernel}
An expected product kernel is also known as a set kernel. 
Let $\mu_{r^{(l)}}:=\mathbb{E}_{r^{(l)}(a)}k(\cdot,a)$ be the mean
embedding \citep{Smola2007} of the distribution $r^{(l)}$ into RKHS
$\mathcal{H}^{(l)}$ induced by the kernel $k$. Assume $k=k_{\text{gauss}}$
(Gaussian kernel) and assume there are $c$ incoming messages $\mathsf{x}:=(r^{(i)}(a^{(i)}))_{i=1}^{c}$
and $\mathsf{y}:=(s^{(i)}(b^{(i)}))_{i=1}^{c}$. The expected product
kernel $\kappa_{\text{pro}}$ is defined as 
\begin{align*}
\kappa_{\text{pro}}\left(\mathsf{x},\mathsf{y}\right) & :=\left\langle \bigotimes_{l=1}^{c}\mu_{r^{(l)}},\bigotimes_{l=1}^{c}\mu_{s^{(l)}}\right\rangle _{\otimes_{l}\mathcal{H}^{(l)}}\\
 & =\prod_{l=1}^{c}\mathbb{E}_{r^{(l)}(a)}\mathbb{E}_{s^{(l)}(b)}k_{\text{gauss}}^{(l)}\left(a,b\right)\approx\hat{\phi}(\mathsf{x})^{\top}\hat{\phi}(\mathsf{y})
\end{align*}
where $\hat{\phi}(\mathsf{x})^{\top}\hat{\phi}(\mathsf{y})=\prod_{l=1}^{c}\hat{\phi}^{(l)}(r^{(l)})^{\top}\hat{\phi}^{(l)}(s^{(l)})$.
The feature map $\hat{\phi}^{(l)}(r^{(l)})$ can be estimated by applying
the random Fourier features to $k_{\text{gauss }}^{(l)}$and taking
the expectations $\mathbb{E}_{r^{(l)}(a)}\mathbb{E}_{s^{(l)}(b)}$.
The final feature map is $\hat{\phi}(\mathsf{x})=\hat{\phi}^{(1)}(r^{(1)})\circledast\hat{\phi}^{(2)}(r^{(2)})\circledast\cdots\circledast\hat{\phi}^{(c)}(r^{(c)})\in\mathbb{R}^{d^{c}}$where
$\circledast$ denotes a Kronecker product and we assume that $\hat{\phi}^{(l)}\in\mathbb{R}^{d}$
for $l\in\{1,\ldots,c\}$.


\subsection{Expected Product Kernel on Joint Embeddings }

Another way to define a kernel on $\mathsf{x},\mathsf{y}$ is to consider
joint distributions $\mathsf{r}=\prod_{i=1}^{c}r^{(i)}$ and
$\mathsf{s}=\prod_{i=1}^{c}s^{(i)}$ given by the product of all incoming messages, and 
define a kernel on the joint distributions.
One can consider the expected product kernel on the joint distributions as 
%
\begin{equation*}
\kappa_{\text{joint}}(\mathsf{x},\mathsf{y}):=\left\langle \mu_{\mathsf{r}},\mu_{\mathsf{s}}\right\rangle _{\mathcal{G}} = \mathbb{E}_{\mathsf{r}(x)} \mathbb{E}_{\mathsf{s}(y)} k(x, y)
\end{equation*}
%
where $\mathcal{G}$ is an RKHS consisting of functions $g:\mathcal{X}^{(1)}\times\cdots\times\mathcal{X}^{(c)}\rightarrow\mathbb{R}$
and $\mathcal{X}^{(l)}$ denotes the domain of $r^{(l)}$ and $s^{(l)}$.

\subsection{Gaussian Kernel on Joint Embeddings }

Given two distributions $\mathsf{r}$ and $\mathsf{s}$, a Gaussian kernel on mean embeddings
\cite{Christmann2010} is defined as 
\[
\kappa_{\text{gauss}}(\mathsf{r}, \mathsf{s}) = \exp\left(-\frac{\|\mu_{\mathsf{r}}-\mu_{\mathsf{s}}\|_{\mathcal{H}}^{2}}{2\gamma^{2}}\right).
\]
This kernel has two parameters, one for $k$ in $\mu_{\mathsf{r}} = \mathbb{E}_{\mathsf{r}}k(x,\cdot)$
and $\gamma^{2}$. Empirically, this kernel performs significantly
better than other kernels on predicting outgoing messages from incoming messages.

\wjnote{Since the norm squared inside exp is nothing but MMD$^2$ in a unit-ball RKHS (Gaussian kernel),
Arthur may have some good intuition why it works well in practice ? 
Would be nice to have a paragraph on theoretical explanation if any.}

\paragraph{Random features for Gaussian kernel on mean embeddings}
\todo{WJ: Change p, q to $\mathsf{r}, \mathsf{s}$}
One way to get random features for the kernel is given
by a two-staged approximation procedure as follows.
By expanding the square in $\kappa_{\text{gauss}}$, we have
%
\begin{align*}
 %& \kappa_{\text{gauss}}(p,q) \\
  \exp\left(-\frac{1}{2\gamma^{2}}\left\langle \mu_{p},\mu_{p}\right\rangle +\frac{1}{\gamma^{2}}\left\langle \mu_{p},\mu_{q}\right\rangle -\frac{1}{2\gamma^{2}}\left\langle \mu_{q},\mu_{q}\right\rangle \right).
\end{align*}
%
Assume $k$ is translation invariant. The inner product of the mean
embeddings $\left\langle \mu_{p},\mu_{q}\right\rangle =\mathbb{E}_{p(x)}\mathbb{E}_{q(y)}k(x-y)\approx\hat{\phi}(p)^{\top}\hat{\phi}(q)$
can be approximated as in Sec. \ref{sub:Expected-Product-Kernel}.
Assume $\hat{\phi}(p),\hat{\phi}(q)\in\mathbb{R}^{D_{in}}$. With
the approximation of $\left\langle \mu_{p},\mu_{q}\right\rangle $,
we have
\[
\kappa_{\text{gauss}}(p,q)\approx\exp\left(-\frac{\|\hat{\phi}(p)-\hat{\phi}(q)\|_{D_{in}}^{2}}{2\gamma^{2}}\right)
%
%:=k_{\text{gauss}}\left(\hat{\phi}(p)-\hat{\phi}(q);\gamma^{2}\right)
\]
which can be thought of as a standard Gaussian kernel on $\mathbb{R}^{D_{in}}$.
We can thus further approximate the Gaussian kernel with
%
$ k_{\text{gauss}}\left(\hat{\phi}(p), \hat{\phi}(q);\gamma^{2}\right)\approx\hat{\psi}(p)^{\top}\hat{\psi}(q)$
%
by standard random Fourier features \citep{Rahimi2007} where $\hat{\psi}(p)\in\mathbb{R}^{D_{out}}$. 
A pseudocode for generating the random features is summarized in Algorithm~\ref{algo:random_features_kgg}.
We need to pre-compute $\left\{ \omega_{i}\right\} _{i=1}^{D_{in}},\left\{ b_{i}\right\} _{i=1}^{D_{in}},\left\{ \nu_{i}\right\} _{i=1}^{D_{out}}$
and $\left\{ c_{i}\right\} _{i=1}^{D_{out}}$ where $D_{in}$ and
$D_{out}$ are the number of random features to be specified. 
A more efficient way to support a large number of random features 
is to store only the random seed used to generate 
them. The coefficients are generated on the fly when needed \citep{Dai2014}. 

\begin{algorithm}[t]
\caption{Construction of two-staged random features for a Gaussian kernel on mean embeddings}
\label{algo:random_features_kgg}
\begin{algorithmic}[1]
\REQUIRE Input distribution $\mathsf{r}$, Fourier transform $\hat{k}$ of the embedding kernel $k$, number of inner features $D_{in}$, number of outer features $D_{out}$, outer Gaussian width $\gamma^2$.
\ENSURE Random features $\hat{\psi}(\mathsf{r}) \in \mathbb{R}^{D_{out}}$.

%\STATE Compute the Fourier transform $\hat{k}$ of the kernel $k$.
\STATE Sample  $\{ \omega_i \}_{i=1}^{D_{in}} \overset{i.i.d}{\sim} \hat{k}$.
\STATE Sample $\{b_i\}_{i=1}^{D_{in}} \overset{i.i.d}{\sim} \text{Uniform}[0, 2\pi] $.
\STATE $\hat{\phi}(\mathsf{r}) = \sqrt{\frac{2}{D_{in}}} \left( \mathbb{E}_{\mathsf{r(x)}} \cos(\omega_{i}^{\top}x+b_{i} ) \right)_{i=1}^{D_{in}} \in \mathbb{R}^{D_{in}}$ \\
%\STATE  $\hat{\phi}(p)=\mathbb{E}_{p(x)}\sqrt{\frac{2}{D_{in}}}\left(\cos\left(\omega_{1}^{\top}x+b_{1}\right),\ldots,\cos\left(\omega_{D_{in}}^{\top}x+b_{D_{in}}\right)\right)^{\top}$.
If $\mathsf{r}(x)=\mathcal{N}(x;m, V )$, 
\small
\begin{equation*}
\hat{\phi}( \mathsf{r}) = \sqrt{\frac{2}{D_{in}}} \left( \cos(w_{i}^{\top}m +b_{i}) \exp \left(-\frac{1}{2}w_{i}^{\top}V w_{i} \right) \right)_{i=1}^{D_{in}}.
\end{equation*}
%Even if $p$ is not a normal distribution, we may still use it as an approximation.
%
\STATE Sample $\{ \nu_i \}_{i=1}^{D_{out}} \overset{i.i.d}{\sim} \hat{k}_{\text{gauss}}(\gamma^{2})$.  
i.e., Fourier transform of a Gaussian kernel with width $\gamma^2$.
\STATE Sample $\{c_i\}_{i=1}^{D_{out}} \overset{i.i.d}{\sim} \text{Uniform}[0, 2\pi] $.
\STATE $\hat{\psi}(\mathsf{r}) = \sqrt{\frac{2}{D_{out}}} \left(  \cos(\nu_{i}^{\top}x + c_{i} ) \right)_{i=1}^{D_{out}} \in \mathbb{R}^{D_{out}}$
\end{algorithmic}
\end{algorithm}


%%%%%%%%%%%%%%


\section{Experiments  \label{sec:Experiments}}

To ensure that the operator is capable of learning the relationship 
of incoming and outgoing messages, we first consider a batch learning setting.

\begin{figure}
\centering
%\missingfigure{Factor graph for binary logistic regression. Just imagine by yourself for now.}
%\includegraphics[scale=0.8,page=2,clip,trim=8cm 18.5cm 8cm 1cm]{img/heess_passing_ep_supp.pdf}
\includegraphics[scale=0.8,page=1,clip,trim=6.5cm 7.5cm 7cm 17.8cm]{img/eslami_jit_supp.pdf}
\caption{Factor graph for binary logistic regression. 
%The logistic factor is highlighted in red. 
%We set $\boldsymbol{\mu}:=\boldsymbol{0}$ and $\boldsymbol{\Lambda}:=I$. 
%\wjnote{This figure will be updated later.}
}
\label{fig:factor_graph_binlog}
\end{figure}

\subsection{Predictive Power}

\paragraph{Logistic} 
As in \cite{Heess2013, Eslami2014}, we consider the logistic factor 
\wjnote{This x is not x in the factor graph. Fix this.}
%
\begin{equation*}
f(z|x)=\delta\left(z-\frac{1}{1+\exp(-x)}\right)
\end{equation*}
%
where $\delta$ is the Dirac delta function, in the context of 
a binary logistic regression model as shown in \figref{fig:factor_graph_binlog}.
%The factor is deterministic when conditioning on $x$. 
The two incoming messages are $m_{X\rightarrow f}(x)=\mathcal{N}(x;\mu,\sigma^{2})$
and $m_{Z\rightarrow f}(z)=\text{Beta}(z;\alpha,\beta)$. 
%\aenote{AE: I think at this point you need to properly define the logistic regression model otherwise incoming messages will be incomprehensible.} 
Following \cite{Eslami2014}, we draw $x, w \sim \mathcal{N}(0, I_{20})$, and set the number of observations to 400.
We repeat the experiment 20 times where in each trial a new $w$ and a new set of observations are generated.
In each trial we collect messages in only the first five iterations of EP from Infer.NET's default implementation of the logistic factor.
All collected messages are randomly partitioned into 5000 training messages and 2000 testing messages.
We learn a message operator using a Gaussian kernel on joint embeddings as described in Eq. .... for sending $m_{f\rightarrow X}$.
Regularization and kernel parameters are chosen by leave-one-out cross validation. 
The number of features are set to $D_{in}=500$ and $D_{out}=1000$. \aenote{AE: Why 1000?}. 
We report $\log KL[q\|\hat{q}]$ where $q=q_{f\rightarrow X}$
is the ground truth output message obtained from Infer.NET's 
and $\hat{q}$ is the message output from the operator. 
For better numerical scaling, regression outputs are set to $(\mathbb{E}_{q}\left[x\right],\log\mathbb{V}_{q}\left[x\right])$ instead of the expectations of the first two moments. 
\aenote{AE: No need to mention log-scaling?} 
The histogram of log KL errors is shown on the left of \figref{fig:Log-KL-divergence-on}.
The right figure shows sample output messages at different log KL errors. 
It can be seen that the operator is able to capture the relationship
of incoming and outgoing messages. 
\wjnote{some more explanation here.}

\begin{figure}
\begin{centering}
\includegraphics[width=6cm]{img/kl_ntr5000_iter5_sf1_st20_fm_joint-crop}
\par\end{centering}

\begin{centering}
\includegraphics[width=6cm]{img/4plots_ntr5000_iter5_sf1_st20_fm_joint-crop.pdf}
\par\end{centering}

\protect\caption{KL-divergence on a logistic factor test set using kernel on joint
embeddings.\label{fig:Log-KL-divergence-on}}
\end{figure}


\paragraph{Compound Gamma}
\wjnote{Hopefully I will have enough time to do this.}
Andrew Gelman \citep{Gelman2006}: ``Gamma prior on precision
is not good''.

\subsection{Uncertainty Estimate}
In this experiment, we aim to verify that the predictive variance of the 
kernel-based message operator is a reliable quantity to be used as a criterion
for deciding whether or not to query the oracle.
We compute the predictive variance using the same operator as used in the logistic factor experiment on the training and test sets.
A plots of KL-divergence errors versus predictive variances for predicting the mean of $m_{f \rightarrow X}$ is shown in \figref{fig:logistic_predvar_g}
% and the variance for predicting $\log(\alpha)$ of $m_{f \rightarrow Z}$ is shown in \figref{fig:logistic_predvar_b}. 



\wjnote{Feel free to fill in more explanation.} 
The uncertainty estimates show a positive correlation with the KL-divergence errors. 
No points lie at the bottom right i.e., high confidence when it makes a big error.
No points lie at the top left i.e., not confident when it should not be.
The fact that the errors on the training set are roughly the same as the errors on the test set
indicates that the operator does not overfit.

\paragraph{Unexplored Region}
One of the most important tasks in JIT learning is to correctly estimate 
the predictive uncertainty in unexplored regions in the space of incoming messages. 
In this section, we show that the operator reports a high
uncertainty when encountered with incoming messages from a region far 
away from the messages in the training set. 
The first figure in \figref{fig:logistic_predvar_unexplored} shows the training set 
as used in the logistic factor experiment. Each point represents one incoming message
from $X$ i.e., $m_{X \rightarrow f}$. The incoming messages from $Z$ are not shown.
We consider two test sets represented as two curves in the first figure. 
The second figure in \figref{fig:logistic_predvar_unexplored} shows predict variance
of each point in the two test sets, where we fix $m_{Z \rightarrow f} := \text{Beta}(z; 1, 2)$.

\wjnote{Finish this explanation. Two key observations. 1. As a test point lies further away
from the training set, the predictive variance grows exponentially.
2. In both test sets, the operator is most certain around zeros mean. 
However, the first test set passes through a lower density region, the operator 
reports a higher uncertainty compared to the second test set. }

Uncertainty estimates from two commonly used random forests on the 
same problem are shown in \figref{fig:tree_uncertainty_unexplored} where
\figref{fig:breiman_uncertainty} shows uncertainty estimates of Breiman's 
random forests \citep{Breiman2001} and \figref{fig:ert_uncertainty} shows 
uncertainty estimates of extremely randomized trees \citep{Geurts2006}.
We set the number of trees to 100 in both cases and other parameters to 
the default settings as set in Scikit-learn package \citep{Pedregosa2011}. 

\wjnote{Need to add how they compute the uncertainty estimates. ?}
\wjnote{Add comments on uncertainty estimates of the trees.}

\begin{figure}
\centering
\includegraphics[width=\columnwidth]{img/uncertainty/logistic_uncertainty-crop}
\caption{Collected training messages for the logistic factor and uncertainty 
estimates of the proposed method on the two uncertainty test sets.}
\label{fig:logistic_predvar_unexplored}
\end{figure}


\begin{figure}[t]
	\centering

	\subfloat[Breiman's random forests\label{fig:breiman_uncertainty}]{
	\includegraphics[width=0.49\columnwidth]{img/uncertainty/rf-n_trees-100-crop}
	}
	%
	\subfloat[Extremely randomized trees\label{fig:ert_uncertainty}]{
	\includegraphics[width=0.49\columnwidth]{img/uncertainty/ert-n_trees-100-crop}
	}
	\caption{Uncertainty estimates of random forests on two uncertainty test sets 
	shown in \figref{fig:logistic_predvar_unexplored}. }
	\label{fig:tree_uncertainty_unexplored}
\end{figure}

%\begin{figure}
%\centering
%\includegraphics[width=0.49\columnwidth]{img/uncertainty/rf-n_trees-100-crop}
%\includegraphics[width=0.49\columnwidth]{img/uncertainty/ert-n_trees-100-crop}
%\caption{Log KL-divergence on a logistic factor test set using kernel on joint embeddings.}
%\label{fig:tree_uncertainty_unexplored}
%\end{figure}

\subsection{Online Learning}
\figref{fig:logistic_temporal_uncertainty} shows predictive variance of KJIT 
on predicting the mean of each $m_{f \rightarrow X}$. The black dotted lines
mark the start of a new inference problem. We set the number of observations 
in the logistic regression problem to 300. It can be seen that the uncertainty
estimate exhibits a periodic structure, repeating itself for every 300 time 
steps. The predictive variance levels off after the operator sees roughly 1000
incoming messages (roughly one-third of the total messages in the first problem).

\figref{fig:logistic_01_loss} shows binary classification errors obtained by 
using the inferred posterior mean parameter on a test set of size 10000 generated 
from the true parameter vector. The loss of KJIT matches that of the importance 
samplier and Infer.NET. \figref{fig:logistic_inference_time} shows the inference 
time spent by all methods in each problem.

% logistic temporal uncertainty
\begin{figure*}[t]
\centering
\includegraphics[width=0.95\textwidth]{img/online/logistic_temporal_uncertainty-crop}
\caption{Uncertainty estimate of KJIT for all incoming messages at each time point 
in the binary logistic regression problem.
\label{fig:logistic_temporal_uncertainty}
}
\end{figure*}

% logistic classification error. 
% Inference time
\begin{figure}[t]
	\centering
	\subfloat[Binary classification error\label{fig:logistic_01_loss}]{
	\includegraphics[width=0.49\columnwidth]{img/online/logistic_01_loss-crop}
	}
	%
	\subfloat[Inference time\label{fig:logistic_inference_time}]{
	\includegraphics[width=0.49\columnwidth]{img/online/logistic_inference_time-crop}
	}
	\caption{Classification performance and inference times of all methods. }
	\label{fig:logistic_performance}
\end{figure}

% compound gamma temporal uncertainty
\begin{figure*}[t]
\centering
\includegraphics[width=0.95\textwidth]{img/online/cg_temporal_uncertainty-crop}
\caption{Uncertainty estimate of KJIT for all incoming messages at each time point 
in the compound Gamma problem.
\label{fig:cg_temporal_uncertainty}
}
\end{figure*}

% compound gamma inferred results and time.
\begin{figure}[t]
	\centering
% 	\subfloat[Posteriors]{
% 	\includegraphics[width=0.49\columnwidth]{img/online/cg_post_corr-crop}
% 	%\missingfigure[figwidth=0.49\columnwidth]{}
% 	}
	%
	\subfloat[Inferred shape]{
	\includegraphics[width=0.31\columnwidth]{img/online/cg_post_shape-crop}
	%\missingfigure[figwidth=0.49\columnwidth]{}
	}
	%
	\subfloat[Inferred shape]{
	\includegraphics[width=0.31\columnwidth]{img/online/cg_post_rate-crop}
	%\missingfigure[figwidth=0.49\columnwidth]{}
	}
	%
	\subfloat[Inference time\label{fig:cg_inference_time}]{
	\includegraphics[width=0.34\columnwidth]{img/online/cg_inference_time-crop}
	}
	\caption{Shape (a) and rate (b) parameters of the inferred posteriors in 
	the compound gamma problem. 
	(c) KJIT is able to infer equally good posterior parameters compared to Infer.NET 
	while requiring several orders of magnitude less runtime.}
	\label{fig:cg_performance}
\end{figure}

% UCI datasets. temporal uncertainty.
\begin{figure*}[t]
\centering
\includegraphics[width=0.95\textwidth]{img/online/uci_temporal_uncertainty-crop}
\caption{Uncertainty estimate of KJIT for all incoming messages on the four UCI datasets.
\label{fig:uci_temporal_uncertainty}
}
\end{figure*}


% UCI data classification error
% Inference time
\begin{figure}[t]
	\centering
	\subfloat[Binary classification error\label{fig:uci_01_loss}]{
	\includegraphics[width=0.52\columnwidth]{img/online/uci_classification-crop}
	}
	%
	\subfloat[Inference time\label{fig:uci_inference_time}]{
	\includegraphics[width=0.47\columnwidth]{img/online/uci_infer_time-crop}
	}
	\caption{Classification performance and inference times on the four UCI datasets. }
	\label{fig:uci_performance}
\end{figure}


% \hrulefill{}
% \hl{Experiments to have}
% \begin{itemize}
% \item For predictive variance on unseen region, we may need two plots where
% in one case the training samples are concentrated in one part of the
% space, and in the second case the training samples are concentrated
% in another part. This is to ensure that the uncertainty estimate does
% not depend on the absolute location.

% \item To determine thresholds, compute the median of log predictive variance 
% on the test set split from the initial minibatch. 

%\item Choose the number of particles in the oracle importance sampler so
%that it takes roughly the same time as our operator. Compute the predictive
%quality of the two. 

% \item As in Fig 3.c of Ali's paper, show that the inference error of sampling
% + JIT is better than sampling alone. This asserts that JIT interpolates
% in the message space in a useful way. \aenote{AE: This would be nice but 
% is probably not critical. Could be one of the last experiments you run?}

% \end{itemize}


%\subsection*{Acknowledgement}
%Gatsby Charitable Foundation for funding.

\section{Conclusions and Future Work\label{sec:Conclusions-and-Future} }

We propose to learn to send EP messages with kernel ridge regression
by casting the KL minimization problem as a supervised learning problem.
With random features, incoming messages to a learned operator are
converted to a finite-dimensional vector. Computing an outgoing message
amounts to computing the moment parameters by multiplying the vector
with a matrix given by the solution of the primal ridge regression.

% future work: a good way to choose kernel parameters

\bibliographystyle{abbrvnat}
\bibliography{ref}


%%%%%%%%%%%%%%%%%%%% Appendix %%%%%%%%%%%%%%%%%%%
\newpage
\onecolumn
\appendix

% \begin{center}
% \textbf{\textcolor{black}{\LARGE{}Supplementary Materials }}
% \par\end{center}{\LARGE \par}
%=========================================
\section*{Supplementary Materials}
%=========================================
\todo[inline]{The following incoherent content needs to be rewritten. I just copied it from my notes. Some part will be moved to the main text.  }

% \begin{figure}[t]
%   \centering
% 
%   %DNormalLogVarBuilder means regressing to Gaussian mean and $\log$ variance.
%   \subfloat[Predicting the mean of $m_{f \rightarrow X}$ (normal distribution)
%   \label{fig:logistic_predvar_g}]{
%   \includegraphics[width=0.7\columnwidth]{img/pred_var/fm_kgg_joint_-bw-out1-crop}
%   }
%   %\includegraphicse[width=7cm]{img/pred_var/fm_kgg_joint_-bw-out2-crop}
%   
%   \subfloat[Predicting $\log(\alpha)$ of $m_{f \rightarrow Z} = \text{Beta}(z; \alpha, \beta)$
%   \label{fig:logistic_predvar_b}]{
%   \includegraphics[width=0.7\columnwidth]{img/pred_var/fm_kgg_joint_-fw-out1-crop}
%   }
%   %\includegraphics[width=7cm]{img/pred_var/fm_kgg_joint_-fw-out2-crop} 	
%   \caption{KL-divergence errors versus predictive variances in the logistic factor problem.}
%   \label{fig:logistic_predvar}
% \end{figure}
  
  
\begin{figure}[h]
  \centering

  \includegraphics[width=5cm]{img/pred_var/fm_kgg_joint_-bw-out1-crop}
  
  \caption{KL-divergence errors versus predictive variances for predicting the 
  mean of $m_{f \rightarrow X}$ (normal distribution) in the logistic factor problem. }
  \label{fig:logistic_predvar_g}
\end{figure}
  

%=========================================  
\section{Performance of Different Kernels}
%=========================================

There are a number of kernels on distributions we may use for just-in-time learning.
To find the most suitable kernel, we compare the performance of each on a 
collection of incoming and output
messages at the logistic factor in the binary logistic regression 
All messages are collected by running EP 20 times on generated toy data. 
Only messages in the first five iterations are considered as messages passed in 
the early phase of EP vary more than in a near-convergence phase.
The output to the regression is the belief messages i.e., the numerator of \eqref{eq:msgPassing:EP}. 

A training set of 5000 messages and a test set of 3000 messages are obtained by 
subsampling all the collected messages. 
Where random features (denoted by RF) are used, kernel widths 
and regularization parameters are chosen by leave-one-out cross validation.
To get a good sense of approximation error from the random features, we also 
compare with incomplete Cholesky factorization (denoted by IChol), a widely used 
gram matrix approximation technique. We use repeated hold-out for parameter selection 
and kernel ridge regression in its dual form when the incomplete Cholesky 
factorization is used. 

Let $f$ be the logistic factor and $m_{f\rightarrow i}$ be an outgoing message. 
Let $q_{f\rightarrow i}$ be the groundtruth belief message (numerator) associated with 
$m_{f\rightarrow i}$. The error metric we use is 
$KL[q_{f\rightarrow i}\,||\, \hat{q}_{f\rightarrow i}]$
where  $\hat{q}_{f\rightarrow i}$ is the belief messages estimated by a learned 
regression function. The following table reports the mean of the log KL-divergence 
and standard deviations.

\begin{tabular}{|l|l|l|}
\hline
& \textbf{mean log KL} & \textbf{s.d. of log KL}  \\\hline
 RF + MV Kernel & -6.9554 & 1.6726 \\\hline
 {RF + Expected product kernel on joint} & -2.7765 & 1.8261  \\\hline
 {RF + Sum of expected product kernels} & -1.0518 & 1.9315  \\\hline
 {RF + Product of expected product kernels} & -2.641 & 1.645  \\\hline
 \textbf{RF + Gaussian kernel on joint embeddings} & -8.9591 & 1.6218  \\\hline
 {IChol + sum of Gaussian kernel on embeddings} & -2.751 & 2.8382  \\\hline
 \textbf{IChol + Gaussian kernel on joint embeddings} & -8.7144 & 1.6864  \\\hline
\end{tabular}

The MV kernel is defined in \secref{sec:mv_kernel}. 
Here product (sum) of expected product kernels refers to a product (sum) of kernels, each 
is an expected product kernel defined on one incoming message. Evidently, the Gaussian 
kernel on joint mean embeddings performs significantly better than other kernels. 


\begin{comment} 
\textbf{Result paths:} 
\begin{itemize}
\item{ \small\verb|exp7/RFGMVMapperLearner_binlogis_bw_proj_n400_iter5_sf1_st20_ntr5000.mat| } 
\item{ \small\verb|exp7/RFGJointEProdLearner_binlogis_bw_proj_n400_iter5_sf1_st20_ntr5000.mat| } 
\item{ \small\verb|exp7/RFGSumEProdLearner_binlogis_bw_proj_n400_iter5_sf1_st20_ntr5000.mat| } 
\item{ \small\verb|exp7/RFGProductEProdLearner_binlogis_bw_proj_n400_iter5_sf1_st20_ntr5000.mat| } 
\item{ \small\verb|exp7/RFGJointKGGLearner_binlogis_bw_proj_n400_iter5_sf1_st20_ntr5000.mat| } 
\item{ \small\verb|exp7/ICholKProduct_binlogis_bw_proj_n400_iter5_sf1_st20_ntr5000.mat| } 
\item{ \small\verb|exp7/ICholKProduct_binlogis_bw_proj_n400_iter5_sf1_st20_ntr5000.mat| } 
\item{ \small\verb|exp7/ICholKSum_binlogis_bw_proj_n400_iter5_sf1_st20_ntr5000.mat| } 
\item{ \small\verb|exp7/ICholKGGaussianJoint_binlogis_bw_proj_n400_iter5_sf1_st20_ntr5000.mat| } 
\end{itemize}

 
\textbf{Selected parameters:} 
\begin{itemize}
\item{ \footnotesize\verb|GenericMapper(CondFMFiniteOut(RandFourierGaussMVMap(mw2s=[1.11211      65.6983], vw2s=[0.001     0.52568])), DNormalLogVarBuilder)| } 
\item{ \footnotesize\verb|GenericMapper(CondFMFiniteOut(RFGJointEProdMap(gw2s=[0.11111      4.2764])), DNormalLogVarBuilder)| } 
\item{ \footnotesize\verb|GenericMapper(CondFMFiniteOut(RFGSumEProdMap(gw2s=[0.555556      21.3243])), DNormalLogVarBuilder)| } 
\item{ \footnotesize\verb|GenericMapper(CondFMFiniteOut(RFGProductEProdMap(gw2s=[0.011111     0.44049])), DNormalLogVarBuilder)| } 
\item{ \footnotesize\verb|GenericMapper(StackInstancesMapper(BayesLinRegFM(RFGJointKGG(embed_w2=[0.055556      2.1709], outer_w2=0.26)), BayesLinRegFM(RFGJointKGG(embed_w2=[0.055556      2.1709], outer_w2=0.10))), DNormalLogVarBuilder)| } 
\item{ \footnotesize\verb|GenericMapper(StackInstancesMapper(CondCholFiniteOut(r=72), CondCholFiniteOut(r=34)), DNormalLogVarBuilder)| } 
\item{ \footnotesize\verb|GenericMapper(StackInstancesMapper(CondCholFiniteOut(r=72), CondCholFiniteOut(r=34)), DNormalLogVarBuilder)| } 
\item{ \footnotesize\verb|GenericMapper(StackInstancesMapper(CondCholFiniteOut(r=90), CondCholFiniteOut(r=119)), DNormalLogVarBuilder)| } 
\item{ \footnotesize\verb|GenericMapper(StackInstancesMapper(CondCholFiniteOut(r=158), CondCholFiniteOut(r=225)), DNormalLogVarBuilder)| } 
\end{itemize}

ans =

   -6.9554    1.6726       NaN
   -2.7765    1.8261       NaN
   -1.0518    1.9315       NaN
   -2.6410    1.6450       NaN
   -8.9591    1.6218       NaN
   -3.2097    1.9868       NaN
   -3.2097    1.9868       NaN
   -2.7510    2.8382       NaN
   -8.7144    1.6864       NaN
\end{comment}

%=========================================
\section{Kernel Ridge Regression}
%=========================================
\subsection{Kernel Ridge Regression in Primal Form}

In our work, we can apply random features to the kernel in the conditional
mean embedding operator. With $\hat{\phi}$, an estimate of an operator
in primal solution is
\begin{align*}
\hat{C}_{x'|w} & =X'W^{\top}\left(WW^{\top}+\lambda I\right)^{-1}\\
 & \approx\underbrace{X'}_{d_{x}\times N}\underbrace{\hat{\Phi}^{\top}}_{N\times D}\underbrace{\left(\hat{\Phi}\hat{\Phi}^{\top}+\lambda I\right)^{-1}}_{D\times D}
\end{align*}
where $\hat{\Phi}=\left(\hat{\phi}(w_{1})|\cdots|\hat{\phi}(w_{N})\right)\in\mathbb{R}^{D\times N}$
and $W=X\otimes Y$ . So an estimated operator can be represented
with a $d_{x}\times D$ matrix where $d_{x}$ is the dimension of
the target sufficient statistic. Applying the operator to a test message
$m_{w\rightarrow f}^{*}$ with random features given by $\phi^{*}\in\mathbb{R}^{D}$
requires just a matrix vector multiplication whose complexity does
not depend on $N$ (unlike in the dual form).


\subsection{Leave-One-Out Cross Validation with Random Features\label{sub:LOOCV-for-Operator}}

The solution of the kernel ridge regression is given by
\[
\hat{C}_{Y|X}=YX^{\top}\left(XX^{\top}+\lambda I\right)^{-1}
\]
where $Y\in\mathbb{R}^{d_{y}\times n}$ and $d_{y}$ is the number
of sufficient statistic outputs i.e., two for a one-dimensional Gaussian.
With random features, we have $K_{ij}\approx\hat{\phi}_{i}^{\top}\hat{\phi}_{j}:=\hat{\phi}(x_{i})^{\top}\hat{\phi}(x_{j})$
where $\hat{\phi}_{i}\in\mathbb{R}^{d}$ and $d$ is the number of
random features. In our work, $x_{i}$ will be an instance in the
tensor product space of the form $x_{i}=m_{x_{1}\rightarrow f}\otimes m_{x_{2}\rightarrow f}\otimes\cdots$.
So the estimator can be rewritten as
\[
\hat{C}_{Y|X}\approx\underbrace{Y}_{d_{y}\times n}\underbrace{\hat{\Phi}^{\top}}_{n\times d}\underbrace{\left(\hat{\Phi}\hat{\Phi}^{\top}+\lambda I\right)^{-1}}_{d\times d}
\]
where $\hat{\Phi}:=\left(\hat{\phi}_{1}|\cdots|\hat{\phi}_{n}\right)\in\mathbb{R}^{d\times n}$.
The LOOCV error can be written as
\[
E_{LOOCV}:=\frac{1}{n}\sum_{j=1}^{n}\|\hat{C}_{Y|X}^{(j)}\hat{\phi}_{j}-y_{j}\|_{d_{y}}^{2}
\]
where $\hat{C}_{Y|X}^{(j)}$ is the estimator obtained with $j^{th}$
instance removed and $y_{j}\in\mathbb{R}^{d_{y}}$. Following the
same idea as in the ordinary ridge regression, we express $\hat{C}_{Y|X}^{(j)}$
in terms of $\hat{C}_{Y|X}$.
\begin{align*}
\hat{C}_{Y|X}^{(j)} & =\left(Y\hat{\Phi}-y_{j}\phi_{j}^{\top}\right)\left(\hat{\Phi}\hat{\Phi}^{\top}+\lambda I-\hat{\phi}_{j}\hat{\phi}_{j}^{\top}\right)^{-1}\\
 & =\left(Y\hat{\Phi}-y_{j}\phi_{j}^{\top}\right)\left(A^{-1}+\frac{A^{-1}\phi_{j}\phi_{j}^{\top}A^{-1}}{1-\phi_{j}^{\top}A^{-1}\phi_{j}}\right)\\
 & =\hat{C}_{Y|X}+h_{jj}^{-1}YL\phi_{j}\phi_{j}^{\top}A^{-1}-h_{jj}^{-1}y_{j}\phi_{j}^{\top}A^{-1}
\end{align*}
where $h_{ii}:=\left(H\right)_{ii}=1-\phi_{i}^{\top}A^{-1}\phi_{i}$,
$A:=\hat{\Phi}\hat{\Phi}^{\top}+\lambda I$ and $L:=\hat{\Phi}^{\top}A^{-1}$.
Note that with these definitions, we have $\hat{C}_{Y|X}=YL$. By
plugging $\hat{C}_{Y|X}^{(j)}$ back to $E_{LOOCV}$, we have
\begin{align*}
E_{LOOCV} & =\frac{1}{n}\sum_{j=1}^{n}\bigg\|\hat{C}_{Y|X}\hat{\phi}_{j}+h_{jj}^{-1}YL\hat{\phi}_{j}\underbrace{\hat{\phi}_{j}^{\top}A^{-1}\hat{\phi}_{j}}_{1-h_{jj}}-h_{jj}^{-1}y_{j}\underbrace{\hat{\phi}_{j}A^{-1}\hat{\phi}_{j}}_{1-h_{jj}}-y_{j}\bigg\|^{2}\\
 & =\frac{1}{n}\sum_{j=1}^{n}\bigg\|\hat{C}_{Y|X}\hat{\phi}_{j}+\left(h_{jj}^{-1}-1\right)YL\hat{\phi}_{j}-\left(h_{jj}^{-1}-1\right)y_{j}-y_{j}\bigg\|_{d_{y}}^{2}\\
 & =\frac{1}{n}\bigg\|\underbrace{\hat{C}_{Y|X}}_{YL}\hat{\Phi}+YL\hat{\Phi}\left(\tilde{H}^{-1}-I\right)-Y\tilde{H}^{-1}\bigg\|_{F}^{2}\\
 & =\frac{1}{n}\bigg\| YL\hat{\Phi}\tilde{H}^{-1}-Y\tilde{H}^{-1}\bigg\|_{Y}^{2}=\frac{1}{n}\bigg\| Y\left(L\hat{\Phi}-I\right)\tilde{H}^{-1}\bigg\|_{F}^{2}\\
 & =\frac{1}{n}\bigg\| YH\tilde{H}^{-1}\bigg\|_{F}^{2}
\end{align*}
where $H:=I-L\hat{\Phi}=I-\hat{\Phi}^{\top}\left(\hat{\Phi}\hat{\Phi}^{\top}+\lambda I\right)^{-1}\hat{\Phi}\in\mathbb{R}^{n\times n}$,
$\tilde{H}$ is a diagonal matrix with the same diagonal as $H$ and
$\|A\|_{F}^{2}:=\trace\left(A^{\top}A\right)$ is the squared Frobenius
norm.


\paragraph{Computing $E_{LOOCV}$}

In computing $E_{LOOCV}$, one should not form $H\in\mathbb{R}^{n\times n}$
explicitly because $n$ is assumed to be huge. 
\begin{align*}
\bigg\| YH\tilde{H}^{-1}\bigg\|_{F}^{2} & =\trace\left(YH\tilde{H}^{-2}H^{\top}Y^{\top}\right)\\
 & =\trace\left(Y\tilde{H}^{-2}Y^{\top}-Y\tilde{H}^{-2}\hat{\Phi}^{\top}A^{-1}\hat{\Phi}Y^{\top}-Y\hat{\Phi}^{\top}A^{-1}\hat{\Phi}\tilde{H}^{-2}Y^{\top}+Y\hat{\Phi}^{\top}A^{-1}\hat{\Phi}\tilde{H}^{-2}\hat{\Phi}^{\top}A^{-1}\hat{\Phi}Y^{\top}\right)
\end{align*}
Let $E:=Y\hat{\Phi}^{\top}A^{-1}\hat{\Phi}\in\mathbb{R}^{d_{y}\times n},B=Y\tilde{H}^{-1}\in\mathbb{R}^{d_{y}\times n}$
then the last line becomes
\begin{align*}
 & \trace\left(BB^{\top}-B\tilde{H}^{-1}E^{\top}-E\tilde{H}^{-1}B^{\top}+E\tilde{H}^{-2}E^{\top}\right)\\
= & \trace\left(BB^{\top}\right)-2\trace\left(E\tilde{H}^{-1}B^{\top}\right)+\trace\left(E\tilde{H}^{-2}E^{\top}\right).
\end{align*}
 Each term in the trace is of size $d_{y}\times d_{y}$. This does
not involve forming an $n\times n$ matrix. $\tilde{H}$ is the diagonal
of $H$. So, $\tilde{H}=\diag\left(I-\hat{\Phi}^{\top}A^{-1}\hat{\Phi}\right)$.
The cost of $O(d^{3})$ seems to be inevitable where $d$ is expected
to be in the order of thousands. The number is large enough that $d^{3}$
is expensive. In computing $E_{LOOCV}$ for selecting the best parameter
combination, we may reduce $d$. The number of features $d$ is set
back to the desired value during testing. 


\paragraph{Operator representation}

Once the best parameter combination is choosen, we increase $d$ to
the desired value and precompute $\hat{C}_{Y|X}\approx\underbrace{Y}_{d_{y}\times n}\underbrace{\hat{\Phi}^{\top}}_{n\times d}\underbrace{\left(\hat{\Phi}\hat{\Phi}^{\top}+\lambda I\right)^{-1}}_{d\times d}$
for test time. One obvious improvement is to avoid forming $\hat{\Phi}$
explicitly since the memory cost will be $O(dn)$ which can already
be huge. $Y\hat{\Phi}$ can be computed incrementally without the
need to form $\hat{\Phi}$. Memory cost for computing $\hat{\Phi}\hat{\Phi}^{\top}$
can be made lower than $O(dn)$ by computing incrementally even though
the computational complexity remains $O(dn^{2})$ which is still huge.
The total cost for computing the operator is $O(d^{2}n+d^{3})$. Linear
dependency on $n$ is unavoidable. 

%=========================================
\section{Kernels and Random Features}
%=========================================

This section reviews relevant kernels and their random feature representations.


\subsection{Random Features}

This section contains a summary of \citesup{Rahimi2007}'s random
Fourier features for a translation invariant kernel. 

A kernel $k(x,y)=\left\langle \phi(x),\phi(y)\right\rangle $ in general
may correspond to an inner product in an infinite-dimensional space
whose feature map $\phi$ cannot be explicitly computed. In \citesup{Rahimi2007},
methods of computing an approximate feature maps $\hat{\phi}$ were
proposed. The approximate feature maps are such that $k(x,y)\approx\hat{\phi}(x)^{\top}\hat{\phi}(y)$
(with equality in expectation) where $\hat{\phi}\in\mathbb{R}^{D}$
and $D$ is the number of random features. High $D$ yields a better
approximation with higher computational cost. Assume $k(x,y)=k(x-y)$
and $x,y\in\mathbb{R}^{d}$. Random Fourier features $\hat{\phi}(x)\in\mathbb{R}^{D}$
such that $k(x,y)\approx\hat{\phi}(x)^{\top}\hat{\phi}(y)$ are generated
as follows.
\begin{enumerate}
\item Compute the Fourier transform $\hat{k}$ of the kernel $k$: $\hat{k}(\omega)=\frac{1}{2\pi}\int e^{-j\omega^{\top}\delta}k(\delta)\, d\delta$.
For a Gaussian kernel with unit width, $\hat{k}(\omega)=\left(2\pi\right)^{-d/2}e^{-\|\omega\|^{2}/2}$.
\item Draw $D$ i.i.d. samples $\omega_{1},\ldots,\omega_{D}\in\mathbb{R}^{d}$
from $\hat{k}$. 
\item Draw $D$ i.i.d samples $b_{1},\ldots,b_{D}\in\mathbb{R}$ from $U[0,2\pi]$
(uniform distribution).
\item $\hat{\phi}(x)=\sqrt{\frac{2}{D}}\left(\cos\left(\omega_{1}^{\top}x+b_{1}\right),\ldots,\cos\left(\omega_{D}^{\top}x+b_{D}\right)\right)^{\top}\in\mathbb{R}^{D}$
\end{enumerate}

\paragraph{Why it works ?}
\begin{thm}
Bochner's theorem. A continuous kernel $k(x,y)=k(x-y)$ on $\mathbb{R}^{m}$
is positive definite iff $k(\delta)$ is the Fourier transform of
a non-negative measure.
\end{thm}
Furthermore, if a translation invariant kernel $k(\delta)$ is properly
scaled, Bochner's theorem guarantees that its Fourier transform $p(\omega)$
is a proper probability distribution. From this fact, we have 
\[
k(x-y)=\int\hat{k}(\omega)e^{j\omega^{\top}\left(x-y\right)}\, d\omega=\mathbb{E}_{\omega}\left[\eta_{\omega}(x)\eta_{\omega}(y)^{*}\right]
\]
where $\eta_{\omega}(x)=e^{j\omega^{\top}x}$ and $\cdot^{*}$ denotes
the complex conjugate. Since both $p$ and $k$ are real, the complex
exponential contains only the cosine terms. Drawing $d$ samples is
for lowering the variance of the approximation.
\begin{thm}
\label{thm:Separation-of-variables.}Separation of variables. Let
$\hat{f}$ be the Fourier transform of $f$. If $f(x_{1},\ldots,x_{n})=f_{1}(x_{1})\cdots f_{n}(x_{n})$,
then $\hat{f}(\omega_{1},\ldots,\omega_{n})=\prod_{i=1}^{n}\hat{f}_{i}(\omega_{i})$. 
\end{thm}
Theorem \ref{thm:Separation-of-variables.} suggests that the random
Fourier features can be extended to product kernel by drawing $\omega$
independently for each kernel. 


\subsection{MV (Mean-Variance) Kernel \label{sec:mv_kernel}}

Assume there are $c$ incoming messages $\left(p^{(i)}\right)_{i=1}^{c}$
. Assume that
\begin{align*}
\mathbb{E}_{p^{(l)}}\left[x\right] & =m_{l}\\
\mathbb{V}_{p^{(l)}}\left[x\right] & =v_{l}\\
\mathbb{E}_{q^{(l)}}\left[y\right] & =\mu_{l}\\
\mathbb{V}_{q^{(l)}}\left[y\right] & =\sigma_{l}^{2}.
\end{align*}
We use $p^{(l)}(x)$ and $m_{x_{i}\rightarrow f}(x)$ interchageably.
Incoming messages are not necessarily Gaussian. They do have to be
in the exponential family. MV (mean-variance) kernel is a a product
kernel on means and variances. 
\begin{align*}
\kappa\left(\left(p^{(i)}\right)_{i=1}^{c},\left(q^{(i)}\right)_{i=1}^{c}\right) & =\prod_{i=1}^{c}\kappa^{(i)}\left(p^{(i)},q^{(i)}\right)\\
 & =\prod_{i=1}^{c}k\left(\left(m_{i}-\mu_{i}\right)/w_{i}^{m}\right)\prod_{i=1}^{c}k\left(\left(v_{i}-\sigma_{i}^{2}\right)/w_{i}^{v}\right)
\end{align*}
where $k$ is a Gassian kernel with unit width. The kernel $\kappa$
has $P:=\left(w_{1}^{m},\ldots,w_{c}^{m},w_{1}^{v},\ldots,w_{c}^{v}\right)$
as its parameters. With this kernel, we treat messages as finite dimensional
vector. All incoming messages $(q^{(i)})_{i=1}^{c}$ are essentially
represented as $\left(\mu_{1},\ldots,\mu_{c},\sigma_{1}^{2},\ldots,\sigma_{c}^{2}\right)^{\top}$.
This treatment reduces the problem of having distributions as inputs
to the familiar problem of having input points from a Euclidean space.
The random features of \cite{Rahimi2007} can be applied straightforwardly. 


\subsection{Expected Product Kernel\label{sub:Expected-Product-Kernel}}

Let $\mu_{r^{(l)}}:=\mathbb{E}_{r^{(l)}(a)}k(\cdot,a)$ be the mean
embedding \citepsup{Smola2007} of the distribution $r^{(l)}$ into
RKHS $\mathcal{H}^{(l)}$ induced by the kernel $k$. Assume $k=k_{\text{gauss}}$
(Gaussian kernel) and assume there are $c$ incoming messages $\mathsf{x}:=(r^{(i)}(a^{(i)}))_{i=1}^{c}$
and $\mathsf{y}:=(s^{(i)}(b^{(i)}))_{i=1}^{c}$. An expected product
kernel $\kappa_{\text{pro}}$ is defined as 
\begin{align*}
\kappa_{\text{pro}}\left(\mathsf{x},\mathsf{y}\right) & :=\left\langle \bigotimes_{l=1}^{c}\mu_{r^{(l)}},\bigotimes_{l=1}^{c}\mu_{s^{(l)}}\right\rangle _{\otimes_{l}\mathcal{H}^{(l)}}=\prod_{l=1}^{c}\mathbb{E}_{r^{(l)}(a)}\mathbb{E}_{s^{(l)}(b)}k_{\text{gauss}}^{(l)}\left(a,b\right)\approx\hat{\phi}(\mathsf{x})^{\top}\hat{\phi}(\mathsf{y})
\end{align*}
where $\hat{\phi}(\mathsf{x})^{\top}\hat{\phi}(\mathsf{y})=\prod_{l=1}^{c}\hat{\phi}^{(l)}(r^{(l)})^{\top}\hat{\phi}^{(l)}(s^{(l)})$.
The feature map $\hat{\phi}^{(l)}(r^{(l)})$ can be estimated by applying
the random Fourier features to $k_{\text{gauss }}^{(l)}$and taking
the expectations $\mathbb{E}_{r^{(l)}(a)}\mathbb{E}_{s^{(l)}(b)}$.
The final feature map is $\hat{\phi}(\mathsf{x})=\hat{\phi}^{(1)}(r^{(1)})\circledast\hat{\phi}^{(2)}(r^{(2)})\circledast\cdots\circledast\hat{\phi}^{(c)}(r^{(c)})\in\mathbb{R}^{d^{c}}$where
$\circledast$ denotes a Kronecker product and we assume that $\hat{\phi}^{(l)}\in\mathbb{R}^{d}$
for $l\in\{1,\ldots,c\}$. 

We first give some results which will be used to derive the Fourier
features for inner product of mean embeddings.
\begin{lem}
If $b\sim\mathcal{N}(b;0,\sigma^{2})$, then $\mathbb{E}[\cos(b)]=\exp\left(-\frac{1}{2}\sigma^{2}\right)$.\label{lemma:e_cos}\end{lem}
\begin{proof}
We can see this by considering the characteristic function of $x\sim\mathcal{N}(x;\mu,\sigma^{2})$
which is given by $c_{x}(t)$.
\[
c_{x}(t)=\mathbb{E}_{x}\left[\exp\left(itb\right)\right]=\exp\left(itm-\frac{1}{2}\sigma^{2}t^{2}\right)
\]
For $m=0,t=1$, we have 
\[
c_{b}(1)=\mathbb{E}_{b}\left[\exp(ib)\right]=\exp\left(-\frac{1}{2}\sigma^{2}\right)=\mathbb{E}_{b}\left[\cos(b)\right]
\]
where we have $\cos()$ because the imaginary part is set to 0 i.e.,
$i\sin(tb)$ vanishes.
\end{proof}
Given two messages $p(x)=\mathcal{N}(x;m_{p},V_{p})$ and $q(y)=\mathcal{N}(y;m_{q},V_{q})$
($d$-dimensional Gaussian), the expected product kernel is defined
as 
\[
\kappa(p,q)=\left\langle \mu_{p},\mu_{q}\right\rangle _{\mathcal{H}}=\mathbb{E}_{p}\mathbb{E}_{q}k(x-y)
\]
where $\mu_{p}:=\mathbb{E}_{p}k(x,\cdot)$ is the mean embedding of
$p$ and we assume that the kernel $k$ associated with $\mathcal{H}$
is translation invariant i.e.g, $k(x,y)=k(x-y)$. The goal here is
to derive random Fourier features for the expected product kernel.
That is, we aim to find $\hat{\phi}$ such that $\kappa(p,q)\approx\hat{\phi}(p)^{\top}\hat{\phi}(q)$
and $\hat{\phi}\in\mathbb{R}^{D}$. 

From \cite{Rahimi2007} which provides random features for $k(x-y)$,
we immediately have
\begin{align*}
\mathbb{E}_{p}\mathbb{E}_{q}k(x-y) & \approx\mathbb{E}_{p}\mathbb{E}_{q}\frac{2}{D}\sum_{i=1}^{D}\cos\left(w_{i}^{\top}x+b_{i}\right)\cos\left(w_{i}^{\top}y+b_{i}\right)\\
 & =\frac{2}{D}\sum_{i=1}^{D}\mathbb{E}_{p(x)}\cos\left(w_{i}^{\top}x+b_{i}\right)\mathbb{E}_{q(y)}\cos\left(w_{i}^{\top}y+b_{i}\right)
\end{align*}
where $\{w_{i}\}_{i=1}^{D}\sim\hat{k}(w)$ (Fourier transform of $k$)
and $\{b_{i}\}_{i=1}^{D}\sim U\left[0,2\pi\right]$. 

Consider $\mathbb{E}_{p(y)}\cos\left(w_{i}^{\top}x+b_{i}\right)$.
Define $z_{i}=w_{i}^{\top}x+b_{i}$. So $z_{i}\sim\mathcal{N}(z_{i};w_{i}^{\top}m_{p}+b_{i},w_{i}^{\top}V_{p}w_{i})$.
Let $d_{i}\sim\mathcal{N}(0,w_{i}^{\top}V_{p}w_{i})$. Then, $p(d_{i}+w_{i}^{\top}m_{p}+b_{i})=\mathcal{N}(w_{i}^{\top}m_{p}+b_{i},w_{i}^{\top}V_{p}w_{i})$
which is the same distribution as that of $z_{i}$. From these definitions
we have,
\begin{align*}
\mathbb{E}_{p(x)}\cos\left(w_{i}^{\top}x+b_{i}\right) & =\mathbb{E}_{p(z_{i})}\cos(z_{i})\\
 & =\mathbb{E}_{p(d_{i})}\cos\left(d_{i}+w_{i}^{\top}m_{p}+b_{i}\right)\\
 & \overset{(a)}{=}\mathbb{E}_{p(d_{i})}\cos(d_{i})\cos(w_{i}^{\top}m_{p}+b_{i})-\mathbb{E}_{p(d_{i})}\sin(d_{i})\sin(w_{i}^{\top}m_{p}+b_{i})\\
 & \overset{(b)}{=}\cos(w_{i}^{\top}m_{p}+b_{i})\mathbb{E}_{p(d_{i})}\cos(d_{i})\\
 & \overset{(c)}{=}\cos(w_{i}^{\top}m_{p}+b_{i})\exp\left(-\frac{1}{2}w_{i}^{\top}V_{p}w_{i}\right)
\end{align*}
where at $(a)$ we use $\cos(\alpha+\beta)=\cos(\alpha)\cos(\beta)-\sin(\alpha)\sin(\beta)$.
We have $(b)$ because $\sin()$ is an odd function and $\mathbb{E}_{p(d_{i})}\sin(d_{i})=0$.
The last equality $(c)$ follows from Lemma \ref{lemma:e_cos}. It
follows that the random features $\hat{\phi}(p)\in\mathbb{R}^{D}$
are given by
\[
\hat{\phi}(p)=\sqrt{\frac{2}{D}}\left(\begin{array}{c}
\cos(w_{1}^{\top}m_{p}+b_{1})\exp\left(-\frac{1}{2}w_{1}^{\top}V_{p}w_{1}\right)\\
\vdots\\
\cos(w_{D}^{\top}m_{p}+b_{D})\exp\left(-\frac{1}{2}w_{D}^{\top}V_{p}w_{D}\right)
\end{array}\right).
\]


Notice that the translation invariant kernel $k$ (i.e., Laplacian)
plays a role by providing $\hat{k}$ from which $\{w_{i}\}_{i}$ are
drawn. For different types of distributions $p,q$, we only need to
be able to compute $\mathbb{E}_{p(x)}\cos\left(w_{i}^{\top}x+b_{i}\right)$.
With $\hat{\phi}(p)$, we have $\kappa(p,q)\approx\hat{\phi}(p)^{\top}\hat{\phi}(q)$
with equality in expectation.


\paragraph{Analytic expression for Gaussian case}

For reference, if $p,q$ are Gaussians defined as before and $k$
is also Gaussian kernel, an analytic expression is available. Assume
$k(x-y)=\exp\left(-\frac{1}{2}\left(x-y\right)^{\top}\Sigma^{-1}\left(x-y\right)\right)$
where $\Sigma$ is the kernel parameter.
\begin{align*}
\mathbb{E}_{p}\mathbb{E}_{q}k(x-y) & =\sqrt{\frac{\det(D_{pq})}{\det(\Sigma^{-1})}}\exp\left(-\frac{1}{2}\left(m_{p}-m_{q}\right)^{\top}D_{pq}\left(m_{p}-m_{q}\right)\right)\\
D_{pq} & :=\left(V_{p}+V_{q}+\Sigma\right)^{-1}
\end{align*}
This is useful in checking the approximation accuracy of $\hat{\phi}$.


\paragraph{Approximation Quality}

\begin{comment}
See \verb|primalKEGauss.m|.
\end{comment}
{} The following result compares the randomly generated features to
the true kernel matrix using various number of $D$. For each $D$,
we repeat 10 times. Sample size is 1000. The derivation seems to be
correct. Gaussian kernel width $\sigma^{2}:=3$. ``max entry diff''
refers to the maximum entry-wise difference between the true kernel
matrix and the approxmated kernel matrix. Messages are one-dimensional
Gaussians with randomly drawn means and variances.

~

\includegraphics[width=10cm]{img/primal_egauss_sanity-crop}


\subsection{Product Kernel on Mean Embeddings}

The kernel $\kappa$ is defined as a product of kernels on mean embeddings.
For each incoming message $l$, we need one set of random weights
$\{w_{i}^{(l)}\}_{i=1}^{D}\sim\hat{k}^{(l)}$ and $\{b_{i}^{(l)}\}_{i=1}^{D}$.
\begin{align*}
\kappa\left(p,q\right) & =\left\langle \bigotimes_{l=1}^{c}\mu_{p^{(l)}},\bigotimes_{l=1}^{c}\mu_{q^{(l)}}\right\rangle _{\otimes_{l}\mathcal{H}^{(l)}}\\
 & =\prod_{l=1}^{c}\left\langle \mu_{p^{(l)}},\mu_{q^{(l)}}\right\rangle _{\mathcal{H}^{(l)}}\\
 & =\prod_{l=1}^{c}\mathbb{E}_{p^{(l)}(x)}\mathbb{E}_{q^{(l)}(y)}k^{(l)}\left(x-y\right)\\
 & \approx\prod_{l=1}^{c}\hat{\phi}^{(l)}(p^{(l)})^{\top}\hat{\phi}^{(l)}(q^{(l)})\\
 & =\hat{\varphi}(p)^{\top}\hat{\varphi}(q)
\end{align*}
where $\hat{\varphi}(p):=\left(\hat{\phi}_{i_{1}}^{(1)}\times\cdots\hat{\phi}_{i_{c}}^{(c)}\right)_{i_{1}\cdots i_{c}=1}^{c}=\hat{\phi}^{(1)}\circledast\hat{\phi}^{(2)}\circledast\cdots\circledast\hat{\phi}^{(c)}\in\mathbb{R}^{D^{c}}$.
The symbol $\circledast$ denotes a Kronecker product and we assume
that each $\hat{\phi}^{(l)}\in\mathbb{R}^{D}$. Without a special
solver which can take advantage of the Kronecker product in $\hat{\varphi}$,
this random feature map might be impractical due to large memory requirement
unless $D$ is very small. 


\subsection{Sum Kernel on Mean Embeddings}

If we instead define $\kappa$ as the sum of $c$ kernels, we have
\begin{align*}
\kappa(p,q) & =\sum_{l=1}^{c}\left\langle \mu_{p^{(l)}},\mu_{q^{(l)}}\right\rangle _{\mathcal{H}^{(l)}}\\
 & \approx\sum_{l=1}^{c}\hat{\phi}^{(l)}(p^{(l)})^{\top}\hat{\phi}^{(l)}(q^{(l)})\\
 & =\hat{\varphi}(p)^{\top}\hat{\varphi}(q)
\end{align*}
where $\hat{\varphi}(p):=\left(\hat{\phi}^{(1)}(p^{(1)})^{\top},\ldots,\hat{\phi}^{(c)}(p^{(c)})^{\top}\right)^{\top}\in\mathbb{R}^{cD}$
whose memory requirement is much lower.




\subsubsection{Kernel on tuples of messages}

As our prediction is from $M_{j}=[m_{V_{i}\rightarrow f}^{j}(v_{i})]_{i=1}^{c}$
($c$ is the number of variables connected to the factor $f$) to
$y_{j}$, in the end, the kernel we need is $\kappa(M_{i},M_{j})$
which is defined on message tuples.
\begin{align*}
\kappa\left(M_{i},M_{j}\right) & =\prod_{l=1}^{c}\kappa_{l}\left(m_{V_{l}\rightarrow f}^{i},m_{V_{l}\rightarrow f}^{j}\right)
\end{align*}
which is a product of Gaussian kernels on mean embedding of each message.
With random features, we have 
\[
\kappa(M_{i},M_{j})\approx\prod_{l=1}^{c}\hat{\psi}_{l}(p_{l}^{i})^{\top}\hat{\psi}_{l}(p_{l}^{j})=\hat{\Psi}(M_{i})^{\top}\hat{\Psi}(M_{j})
\]
where $\hat{\Psi}(M_{i}):=\left(\left(\hat{\psi}_{1}(p_{1}^{i})_{i_{1}}\right)\times\cdots\times\left(\hat{\psi}_{c}(p_{c}^{i})_{i_{c}}\right)\right)_{i_{1}\cdots i_{c}=1}^{c}=\hat{\psi}_{1}(p_{1}^{i})\circledast\hat{\psi}_{2}(p_{2}^{i})\circledast\cdots\circledast\hat{\psi}_{c}(p_{c}^{i})\in\mathbb{R}^{D_{out}^{c}}$.
The symbol $\circledast$ denotes a Kronecker product. The total number
of random features is given by $D=\left(D_{out}\right)^{c}$. It can
be seen that the total number of features grows exponentially in the
number of connected variables. We can also use 
\begin{align*}
\kappa\left(M_{i},M_{j}\right) & =\sum_{l=1}^{c}\kappa_{l}\left(m_{V_{l}\rightarrow f}^{i},m_{V_{l}\rightarrow f}^{j}\right)
\end{align*}
which gives $D=cD_{out}$. We expect the product kernel to be able
to better capture the interaction among incoming messages.

Yet another way to form a kernel on tuples of messages is to consider
a joint embedding. Let $\mathbb{M}_{i}:=\prod_{l=1}^{c}m_{V_{l}\rightarrow f}^{i}(v_{l})$
e.g., a product of all incoming messages. We can treat $\mathbb{M}_{i}$
as a joint distribution of one variable $V:=(v_{1},\ldots,v_{c})$.
A joint embedding kernel is straightforwardly given by defining a
Gaussian kernel on mean embeddings of these joint distributions:
\[
\kappa(M_{i},M_{j})=\kappa_{\text{gauss}}(\mathbb{M}_{i},\mathbb{M}_{j}).
\]
An advantage of this kernel is that the final number of random features
is $D_{out}$ as compared to $D_{out}^{c}$ (product kernel) and $cD_{out}$
(sum kernel). One potential disadvantage is that we treat all neighbouring
variables as independent.


\paragraph{Experiments on Random Features for Gaussian Kernel on Mean Embeddings }

It is unclear how $D_{in}$ and $D_{out}$ affect the quality of the
random feature approximation. We quantify the effect empirically as
follows. We generate 300 Gaussian messages, compute the true gram
matrix and the approximate gram matrix given by the random features,
and report the Frobenius norm of the difference of the two matrices
on a grid of $D_{in}$ and $D_{out}$. For each $(D_{in},D_{out})$,
we repeat 20 times with a different set of random features and report
the averaged Frobenius norm.

\includegraphics[width=8cm]{img/kggauss_rf_grid-crop} 

The result suggests that $D_{out}$ has more effect in improving the
approximation.


\bibliographystylesup{abbrvnat}
\bibliographysup{refappendix} 
\end{document}
